\section[Podsumowanie (Michał Krakowiak, Kacper Połom, Jakub Wyka)]{Podsumowanie}
\label{chapter:summary}
Temat pracy inżynierskiej "
system dedykowany wspomagający testowanie bezpieczeństwa systemu komputerowego" 
jasno określa zawartość tego dokumentu. Skupiła się ona na bezpieczeństwie systemu 
komputerowego uwzględniając wszystkie jego składowe. 
\begin{itemize}
    \item warstwa fizyczna
    \item oprogramowanie
    \item użytkownik
\end{itemize}
Spośród wielu możliwych testów bezpieczeństwa zdecydowaliśmy się wybrać takie, które 
dotyczyć będą wszystkich przedstawionych wyżej aspektów. W tym celu zdecydowaliśmy, że 
zastosujemy sprzętowy komponent w postaci urządzenia wykonującego skrypty testowe. 
Dzięki temu każdy scenariusz testowy sprawdza świadomość użytkownika dotyczącą 
niezaufanych urządzeń podłączanych do stacji roboczej. Następnie wybrany został obszar 
testów bezpieczeństwa wzorowanych na prawdziwych atakach wymierzonych w systemy komputerowe. 
Zdecydowaliśmy się oprzeć ten projekt na testach przejęcia kontroli nad klawiaturą oraz 
wrażliwych danych użytkownika korzystającego z sieci internet.


-- wnioski z 2 rozdziału MK

-- wnioski z 4 rozdziału KP

Przedstawiony system spełnia swoje założenia i z powodzeniem może być wykorzystywany 
do realizacji swojego głównego celu, jakim jest testowanie bezpieczeństwa systemów 
komputerowych. Jego implementacja przebiegła pomyślnie i efektem tego projektu 
inżynierskiego jest działający system, zgodny ze specyfikacją przedstawioną w tej pracy. 

Dalszy rozwój systemu mógłby się opierać na dodaniu nowych scenariuszy testowych 
do już istniejących konfiguracji karty sieciowej oraz klawiatury. 
W związku z tym system umożliwa łatwą rozbudowę o nowe skrypty testujące bezpieczeństwo.
Jeszcze ciekawsze wydaje się dodanie nowej konfiguracji urządzenia wykonującego wykorzystując 
istniejące sterowniki jądra systemu Linux, takie jak kamerka internetowa, drukarka lub 
pamięć masowa.
