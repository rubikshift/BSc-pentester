\section[Podsumowanie (Michał Krakowiak, Kacper Połom, Jakub Wyka)]{Podsumowanie}
\label{chapter:summary}
Temat pracy inżynierskiej \textit{System dedykowany wspomagający testowanie bezpieczeństwa systemu komputerowego}
jasno określa zawartość tego dokumentu. Tekst skupia się na bezpieczeństwie systemu 
komputerowego uwzględniając wszystkie jego składowe. 
\begin{itemize}
    \item warstwę fizyczną
    \item oprogramowanie
    \item użytkownika
\end{itemize}
Spośród wielu możliwych testów bezpieczeństwa wybrano takie, które 
dotyczą wszystkich przedstawionych wyżej aspektów. W tym celu zdecydowano, że 
zastosowany zostanie sprzętowy komponent w postaci urządzenia wykonującego skrypty testowe. 
Dzięki temu każdy scenariusz testowy sprawdza świadomość użytkownika dotyczącą 
niezaufanych urządzeń podłączanych do stacji roboczej. Następnie wybrany został obszar 
testów bezpieczeństwa wzorowanych na prawdziwych atakach wymierzonych w systemy komputerowe. 
Zdecydowano aby oprzeć ten projekt na testach bezpieczeństwa związanych z 
symulowaniem działania klawiatury oraz kradzieży
wrażliwych danych użytkownika korzystającego z sieci internet.


Realizacja projektu wspomagającego ofensywne testy bezpieczeństwa jest wymagająca przez względy prawne.
Wciąż oczekuje się nowych przejrzystszych regulacji dotyczących testów penetracyjnych i programów bug bounty.
Na każdym etapie implementacji należało podkreślać cel pracy, jakim jest pomoc w zabezpieczaniu zlokalizowanych słabych punktów systemu komputerowego.
Ponadto otrzymane rezultaty uświadamiają jakim zagrożeniem stała się możliwość oprogramowywania urządzeń USB.
Brak autoryzacji akcesoriów pozwala wprawdzie na bezproblemowe i niemal natychmiastowe ich działanie, ale nadużycie zaufania wydaje się dosyć dużym zagrożeniem.

-- wnioski z 4 rozdziału KP

Przedstawiony system spełnia założenia projektowe i z powodzeniem może być wykorzystywany 
do realizacji swojego głównego celu, jakim jest testowanie bezpieczeństwa systemów 
komputerowych. Jego implementacja przebiegła pomyślnie. 

Dalszy rozwój systemu mógłby się opierać na dodaniu nowych scenariuszy testowych 
do już istniejących konfiguracji karty sieciowej oraz klawiatury. 
W związku z tym system umożliwia łatwą rozbudowę o nowe skrypty testujące bezpieczeństwo.
Jeszcze ciekawsze wydaje się dodanie nowej konfiguracji urządzenia wykonującego takiego jak: kamerka internetowa, drukarka lub 
pamięć masowa, wykorzystując sterowniki systemu Linux.
