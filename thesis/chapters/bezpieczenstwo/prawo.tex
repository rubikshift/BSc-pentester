\subsection[Kwestie prawne]{Kwestie prawne}
Testowanie bezpieczeństwa systemu informatycznego, w szczególności to o charakterze ofensywnym, polegające na wykryciu podatności, może pozornie przypominać działania nielegalne. 
Warty uwagi jest art. 296b \S 1 Kodeks Karnego: "Kto wytwarza, pozyskuje, zbywa lub udostępnia innym osobom urządzenia lub programy komputerowe przystosowane do popełnienia przestępstwa określonego w art 165 \S 1 pkt 4, art. 267 \S 3, art. 268a \S 1 albo \S 2 w związku z \S 1, art. 296 \S 1 lub 2 albo art. 269a, a także także hasła komputerowe, kody dostępu lub inne dane umożliwiające nieuprawniony dostęp do informacji przechowywanych w systemie informatycznym, systemie teleinformatycznym lub sieci teleinformatycznej, podlega karze pozbawienia wolności od 3 miesięcy do lat 5"~\cite{kk}.
W tym samym artykule wprowadzony jest odpowiedni kontratyp przestępstwa, który zezwala na działania mające na celu zabezpieczenie systemu informatycznego: "Nie popełnia przestępstwa określonego w \S 1, kto działa wyłącznie w celu zabezpieczenia systemu informatycznego, systemu teleinformatycznego lub sieci teleinformatycznej przed popełnieniem przestępstwa wymienionego w tym przepisie albo opracowania metody takiego zabezpieczenia"~\cite{kk}. Polskie prawo jeszcze nie precyzuje w pełni legalności działań podejmowanych w ramach np. programów bug bounty. W związku z opublikowaną przez Ministerstwo Cyfryzacji \textit{"Strategią Cyberbezpieczeństwa Rzeczypospolitej Polskiej na lata 2017-2022"} w najbliższych latach można oczekiwać nowych regulacji w tym obszarze~\cite{strategia_mc}.
