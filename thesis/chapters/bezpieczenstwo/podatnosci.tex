\subsection[Klasyfikacja podatności]{Klasyfikacja podatności}
Obecne w różnych systemach podatności mogą mieć podobną naturę.
Wynikają z podobnych błędów implementacyjnych lub projektowych, pozwalają na wyprowadzenie ataku wg. istniejącego wcześniej schematu itp.
Jednak każdą podatność można uznać jako zaburzenie triady CIA \textit{(Confidentiality - poufność, Integrity - integralność, Availability - dostępność)}.
Istnieją takie projekty jak OWASP \textit{(Open Web Application Security Project)} Top 10, które wyróżniają takie klasy jak:
\begin{itemize}
    \item Wstrzyknięcia
    \item Niepoprawne uwierzytelnianie
    \item Korzystanie z komponentów ze znanymi podatnościami~\cite{owasp}.
\end{itemize}
Można także zaproponować klasyfikację uwzględniającą efekt wykorzystania podatności np.:
\begin{itemize}
    \item RCE \textit(Remote Code Execution) - zdalne wykonanie kodu
    \item DoS \textit(Denial of Service) - odmowa świadczenia usługi
    \item Eskalacja uprawnień.
\end{itemize}
Istnieją nawet słowniki takie jak CVE \textit{(Common Vulnerabilities and Exposures)}, który jest zarządzany przez organizację MITRE.
Każdej podatności przypisuje się numer jak np. CVE-2011-0638, który wskazuje na słabość systemu \textit{Windows} związaną z obsługą nowo przyłączonych urządzeń USB.
MITRE zarządza także systemem NVD \textit{(National Vulnerability Database)}, gdzie można znaleźć np. oceny krytyczności danych podatności.
