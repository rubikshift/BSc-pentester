\subsection[Wybrane zagadnienia dotyczące testów bezpieczeństwa]{Wybrane zagadnienia dotyczące testów bezpieczeństwa}
\subsubsection[Zasób]{Zasób}
Zasobami \textit{(ang. assets)} określa takie elementy otoczenia, które są powiązane z informacjami o ograniczonym dostępie (tzn. dostęp do nich mają tylko osoby uprawnione do ich wglądu lub modyfikacji). Zasobami mogą być np. dane lub urządzenia. \cite{baloch}
\subsubsection[Podatność]{Podatność}
Podatnością \textit{(ang. vulnerability)} określa się pewną słabość w zasobie, która może zostać wykorzystana do uzyskania nieautoryzowanego dostępu. \cite{baloch}
\subsubsection[Zagrożenie]{Zagrożenie}
Zagrożeniem \textit{(ang. threat)} określa się potencjalne niebezpieczeństwo dla systemu komputerowego. Zagrożeniem jest sukcesywne wykorzystanie podatności oraz może nim napastnik próbujący uzyskać nieautoryzowany dostęp do zasobu. \cite{baloch}
\subsubsection[Exploit]{Exploit}
Exploit to rzecz, która ma na celu wykorzystanie pewnej konkretnej podatności zasobu. Wynikiem działania exploitu jest uzyskanie przez napastnika nieautoryzowanego dostępu do pewnych danych.\cite{baloch}  
\subsubsection[Test penetracyjny]{Test penetracyjny}
Test penetracyjny to zestaw działań, które mają na celu zlokalizowanie podatności oraz sprawdzenie możliwości ich wykorzystania \textit{(exploitacji)}. Podejmowane działania ze swojej natury przypominają te przeprowadzane przez rzeczywistego napastnika. W związku z tym o legalności przeprowadzonego testu decyduje zawarcie odpowiednich zgód i umów. Osobą przeprowadzającą testy penetracyjne określa się mianem pentestera. \cite{baloch}
\subsubsection[Bug bounty]{Bug bounty}
Bug bounty to programy, które stanowią pewnego rodzaju pozwolenie od organizatora na poszukiwanie podatności w systemie informatycznym. Badacze, którzy wskażą nowe możliwości exploitacji systemu często mogą liczyć na nagrodę pieniężną. \cite{grayhat}
\subsubsection[Kwestie prawne]{Kwestie prawne}
Testowanie bezpieczeństwa systemu informatycznego, w szczególności to o charakterze ofensywnym, polegające na wykryciu podatności, może pozornie przypominać działania nielegalne. 
Warty uwagi jest art. 296b \S 1 Kodeks Karnego: "Kto wytwarza, pozyskuje, zbywa lub udostępnia innym osobom urządzenia lub programy komputerowe przystosowane do popełnienia przestępstwa określonego w art 165 \S 1 pkt 4, art. 267 \S 3, art. 268a \S 1 albo \S 2 w związku z \S 1, art. 296 \S 1 lub 2 albo art. 269a, a także także hasła komputerowe, kody dostępu lub inne dane umożliwiające nieuprawniony dostęp do informacji przechowywanych w systemie informatycznym, systemie teleinformatycznym lub sieci teleinformatycznej, podlega karze pozbawienia wolności od 3 miesięcy do lat 5." \cite{kk}
W tym samym artykule wprowadzony jest odpowiedni kontratyp przestępstwa, który zezwala na działania mające na celu zabezpieczenie systemu informatycznego: "Nie popełnia przestępstwa określonego w \S 1, kto działa wyłącznie w celu zabezpieczenia systemu informatycznego, systemu teleinformatycznego lub sieci teleinformatycznej przed popełnieniem przestępstwa wymienionego w tym przepisie albo opracowania metody takiego zabezpieczenia.". \cite{kk} Polskie prawo jeszcze nie precyzuje w pełni legalności działań podejmowanych w ramach np. programów bug bounty. W związku z opublikowaną przez Ministerstwo Cyfryzacji \textit{"Strategią Cyberbezpieczeństwa Rzeczypospolitej Polskiej na lata 2017-2022"} w najbliższych latach można oczekiwać nowych regulacji w tym obszarze. \cite{strategia_mc}
