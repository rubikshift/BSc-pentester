\subsection[Wybrane zagadnienia dotyczące testów bezpieczeństwa]{Wybrane zagadnienia dotyczące testów bezpieczeństwa}
\subsubsection[Zasób]{Zasób}
Zasobami \textit{(ang. assets)} określa takie elementy otoczenia, które są powiązane z informacjami o ograniczonym dostępie (tzn. dostęp do nich mają tylko osoby uprawnione do ich wglądu lub modyfikacji). Zasobami mogą być np. dane lub urządzenia~\cite{baloch}.
\subsubsection[Podatność]{Podatność}
Podatnością \textit{(ang. vulnerability)} określa się pewną słabość w zasobie, która może zostać wykorzystana do uzyskania nieautoryzowanego dostępu~\cite{baloch}.
\subsubsection[Zagrożenie]{Zagrożenie}
Zagrożeniem \textit{(ang. threat)} określa się potencjalne niebezpieczeństwo dla systemu komputerowego. Zagrożeniem jest sukcesywne wykorzystanie podatności oraz może nim być napastnik próbujący uzyskać nieautoryzowany dostęp do zasobu~\cite{baloch}.
\subsubsection[Exploit]{Exploit}
Exploit to rzecz, która ma na celu wykorzystanie pewnej konkretnej podatności zasobu. Wynikiem działania exploitu jest uzyskanie przez napastnika nieautoryzowanego dostępu do pewnych danych~\cite{baloch}. 
\subsubsection[Test penetracyjny]{Test penetracyjny}
Test penetracyjny to zestaw działań, które mają na celu zlokalizowanie podatności oraz sprawdzenie możliwości ich wykorzystania. Podejmowane działania ze swojej natury przypominają te przeprowadzane przez rzeczywistego napastnika. W związku z tym o legalności przeprowadzonego testu decyduje zawarcie odpowiednich zgód i umów. Osobę przeprowadzającą testy penetracyjne określa się mianem pentestera~\cite{baloch}.
\subsubsection[Bug bounty]{Bug bounty}
Bug bounty to programy, które stanowią pewnego rodzaju pozwolenie od organizatora na poszukiwanie podatności w danym systemie informatycznym. Badacze, którzy wskażą nowe możliwości exploitacji systemu często mogą liczyć na nagrodę pieniężną~\cite{grayhat}.
\subsubsection[Inżynieria społeczna (socjotechnika) (Jakub Wyka)]{Inżynieria społeczna (socjotechnika)}
Inżynierię społeczną, inaczej nazywaną socjotechniką, można zdefiniować jako 
zbiór technik wykorzystywanych do uzyskiwania informacji lub manipulowania
zachowaniem innych osób. Dokładniejsza definicja, która odpowiada tematyce tej
pracy, mówi o wykorzystaniu tych działań w celu naruszenia bezpieczeństwa systemu komputerowego.
Ważne jest stwierdzenie, mówiące że ataki socjotechniczne są skierowanego w najsłabszy 
punkt zabezpieczenia każdej organizacji - ludzi~\cite{socialeng, mitnick}.
