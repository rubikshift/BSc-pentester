\subsection[Bezpieczeństwo fizycznych interfejsów komputera]{Bezpieczeństwo fizycznych interfejsów komputera}
Znaczna część publikacji dotyczących cyberbezpieczeństwa skupia się na zagrożeniach dotyczących aplikacji internetowych.
Jednak postępująca miniaturyzacja przyczynia się do powstawania złośliwych urządzeń ukrytych w m.in. akcesoriach USB (np. w kablach~\cite{niebezpiecznik}). Takie urządzenia mogą być wykorzystane m.in. do wykonania złośliwego kodu w komputerze, do którego zostały podłączone.
Najsłabszym ogniwem każdego systemu informatycznego jest człowiek~\cite{mitnick}. Przeciętny użytkownik rzadko jest świadomy tego typu zagrożeń. Nie przejmuje się on konsekwencjami korzystania z urządzeń nieznanego pochodzenia.
W rezultacie rośnie ryzyko związane z wykorzystaniem takiego wektora ataku.
