\subsection[Bezpieczeństwo fizycznych interfejsów komputera]{Bezpieczeństwo fizycznych interfejsów komputera}  \label{bez:int}
Podatności standardu USB wynikają z założenia, że podłączonym urządzeniom można ufać, w najgorszym przypadku działają błędnie, ale nie złośliwie~\cite{malicious_peripherals}.
Współczesne systemy operacyjne implementują mechanizm \textit{plug and play}, który polega na zdolności do niemal natychmiastowej współpracy z rożnego rodzaju urządzeniami.
Nie jest wymagana ingerencja użytkownika w konfigurację.
Urządzenia same raportują systemowi operacyjnemu swoją tożsamość, na podstawie której podejmowana jest decyzja jak je obsłużyć.
Istnieją nawet generyczne klasy jak np. klawiatury, od których użytkownicy wymagają, żeby "po prostu działały".
Problemem w takim mechanizmie działania jest brak jakiejkolwiek weryfikacji.
Dane otrzymane od urządzenia mogą być spreparowane w taki sposób, aby wywołać błąd w sterowniku.
Ponadto istnieje możliwość podszycia się pod inne urządzenia, naruszając oczekiwania użytkownika.
Żaden system nie sprawdza czy tożsamość urządzenia faktycznie odpowiada temu co miało być podłączone.
Dla przykładu istnieją urządzenia, które przypominają kable, ale są w stanie działać jak klawiatura, żeby wykonać niebezpieczny kod w systemie~\cite{niebezpiecznik}. Wykorzystywana jest w tym przypadku podatność oznaczona jako~\textit{CVE-2011-0638}.
Jest to możliwe dzięki coraz niższej cenie zakupu takich produktów oraz możliwości programowej realizacji urządzenia.
Istnieje także czynnik w postaci łatwej dystrybucji i dotarcia do użytkownika np. pod postacią darmowych gadżetów promocyjnych.
Przeciętny użytkownik rzadko jest świadomy tego typu zagrożeń, wobec tego korzystając z akcesoriów nieznanego pochodzenia naraża system komputerowy na atak.
