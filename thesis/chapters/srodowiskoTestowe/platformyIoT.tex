\subsection[Przegląd platform IoT]{Przegląd platform IoT}
\label{subsection:platform}
Jest wiele platform IoT, które bardzo się różnią, dlatego warto przy szukaniu konkretnej przyjąć pewne założenie. Takim założeniem może być wybranie platformy, która nie zależy od hosta ani od jakości wsparcia, oznacza to, że taką platformę możemy uruchomić na dowolnej maszynie. Kolejnym założeniem jest wybranie platformy, która będzie darmowa, te warunki spełniają jedynie platformy z otwartym oprogramowaniem (ang. Open Sorce). Platformy IoT posiadają wiele różnic, mogą inaczej analizować dane (np. w czasie rzeczywistym, lub na prośbę użytkownika). Mogą również w inny sposób wizualizować zebrane dane, są też takie platformy, które w ogóle nie obsługują wizualizacji. Takich różnic jest znacznie więcej, ale lepiej skupić się na najważniejszych aspektach, takich jak możliwość zarządzania platformami, obsługiwane protokoły umożliwiające zbieranie danych, język w jakim została napisana platforma, mechaniki do zapewnienia bezpieczeństwa oraz jakie bazy danych może obsłużyć platforma. 
\begin{table}[H]
    \begin{tabular}{|p{3cm}|p{2cm}|p{6.5cm}|p{1.5cm}|}
        \hline
        \textbf{Rodzaj platformy} & \textbf{Zarządzanie urządzeniami}  & \textbf{Protokoły do zbierania danych} & \textbf{Język} \\
        \hline
        Kaa IOT platform & tak & MQTT, CoAP, XMPP, TCP, HTTP & Java \\
        \hline
        Sitewhere  & tak & MQTT, AMQP, Stomp, WebSocket, DSC & Java \\
        \hline
        ThingSpeak & nie & HTTP & Ruby \\
        \hline
        Mainflux & tak & HTTP, MQTT, WebSocket, CoAP & Go \\
        \hline
        Zetta & nie & HTTP & Javascript \\
        \hline
    \end{tabular}
    \caption{Porównanie platform open sorce }  
    \label{tab:plat1}
\end{table}

\begin{table}[H]
    \begin{tabular}{|p{3cm}|p{6cm}|p{4.5cm}|}
        \hline
        \textbf{Rodzaj platformy} & \textbf{Bezpieczeństwo}  & \textbf{Obsługiwane bazy danych} \\
        \hline
        Kaa IOT platform & Ssl, RSA key 2048 & MongoDB, Cassandra, Hadoop, Oracle NoSQL \\
        \hline
        Sitewhere & Ssl, spring security & MongoDB, HBase , InfluxDB \\
        \hline
        ThingSpeak & Basic Authentication  & MySql \\
        \hline
        Mainflux & JWT encrypted, signed tokens, OAuth2.0, public key infrastructure (PKI) oraz client-side certificates & Cassandra, MongoDB, InfluxDB, PostgreSQL \\
        \hline
        Zetta  & Basic Authentication  & Brak \\
        \hline
    \end{tabular} 
    \caption{Porównanie platform open sorce} 
    \label{tab:plat2}
\end{table}
Zetta oraz ThingSpeak nie wspierają zarządzania urządzeniami oraz jedynym protokołem umożliwiającym przesyłanie danych jest HTTP. Ten protokół zapewnia wysoką gwarancję dostarczenia wiadomości oraz otrzymania odpowiedzi. Ma to jednak swoje minusy, ponieważ jego przepustowość jest znacznie mniejsza od MQTT. Komunikacja przy użyciu takiej technologii będzie zbyt wolna, szczególnie w momencie, gdy pentester będzie równocześnie testował wiele systemów. W odróżnieniu od tych dwóch platform Mainflux, SiteWhere oraz Kaa IoT wspierają takie protokoły, które są w stanie obsłużyć duży ruch, różnica pomiędzy nimi jest w zagwarantowaniu bezpieczeństwa. Mainflux pozwala na użycie Json Web Token (JWT), jest to standard, który określa sposób bezpiecznego przesyłania wiadomości, jako obiekt JSON. Otrzymaną informację można zweryfikować, ponieważ jest podpisana cyfrowo. Taką wiadomość można podpisać z użyciem klucza tajnego (algorytm HMAC) albo za pomocą pary kluczy publiczny / prywatny (algorytm RSA lub ECDSA). Dodatkowo te trzy platformy używają dockera co pozwala na szybkie uruchamianie oraz łatwą dystrybucję.%Docker jest to narzędzie pozwalające na łatwiejsze tworzenie, wdrażanie oraz uruchamianie aplikacji. Umożliwia również korzystanie z tego samego jądra Linuksa, co system, na którym działa. Daje to wysoki wzrost wydajności, oraz zmniejsza rozmiar.
Atutem Mainfluxa jest język w jakim został napisany, dzięki Go rozmiar kontenera jest niewielki, poza tym kod napisany w takim języku jest znacznie szybszy od tych w Javie czy C\#, wynika to z faktu, że Go jest językiem kompilowanym. Kolejną zaletą jest fakt, że język Go jest stosunkowo młody oraz ciągle rozwijany przez firmę google. Dzięki wsparciu dużej korporacji oraz ciągle rosnącej popularności, można mieć pewność, że ten język będzie stawał się coraz lepszy. %//może jeszcze 2 słowa co nam daje fakt że jęzk go jest ciągle rozwijany???
Platforma Mainflux wydaje się być najlepszym wyborem przy tworzeniu ST. Przewagę nad innymi platformami zawdzięcza nie tylko językowi Go, ale również bezpieczeństwu, ponieważ oferuje wiele rozwiązań ochrony przesyłanych informacji.
Silną stroną Mainfluxa jest również dokumentacja, która nie narzuca programiście sposobu implementacji.
%bardzo dobrej dokumentacji. 
%Mainflux również się wyróżnia pod względem 
