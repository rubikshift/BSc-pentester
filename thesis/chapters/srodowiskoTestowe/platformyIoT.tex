\subsection[Przegląd platform IoT]{Przegląd platform IoT}
Wyrażenie Internet Rzeczy pojawiło się już kilka lat temu i od tego czasu znacznie zyskało na popularności. Już teraz wiele firm oferuje nam różnie rozwiązania od termostatu do inteligentnych urządzeń domowych takie jak lodówka, pralka, zmywarka. Żeby zapewnić niezbędne wymagania napisano wiele platform IoT, ciężko jest, więc wybrać pasującą, dlatego warto przed szukaniem konkretnej platformy przyjąć pewne założenie. Naszym założeniem będzie, wybraniem takiej platforma, która nie zależy od hosta ani od jakości wsparcia oraz wybrana platforma musi być darmowa, te warunki spełniają jedynie platformy z otwartym oprogramowaniem (eng. Open Sorce).
\begin{table}[H]
    \begin{tabular}{|p{2cm}|p{2cm}|p{2cm}|p{3cm}|p{4cm}|p{1cm}|}
        \hline
        \textbf{Rodzaj platformy} & \textbf{Zarządzanie urządzeniami}  & \textbf{Bezpieczeństwo}  & \textbf{Protokoły do zbierania danych}  & \textbf{Obsługiwane bazy danych}  & \textbf{Język} \\
        \hline
        Kaa IOT platform & tak & Ssl, RSA key 2048 & MQTT, CoAP, XMPP, TCP, HTTP & MongoDB, Cassandra, Hadoop, Oracle NoSQL & java \\
        \hline
        Sitewhere  & tak & Ssl, spring security & MQTT, AMQP, Stomp, WebSocket, DSC & MongoDB, HBase , InfluxDB & java \\
        \hline
        ThingSpeak & nie & Basic Authentication & HTTP & MySql & rubi \\
        \hline
        Mainflux & tak & JWT encrypted and signed tokens, OAuth2.0, public key infrastructure (PKI) oraz client-side certificates & HTTP, MQTT, WebSocket, CoAP & Cassandra, MongoDB, InfluxDB, PostgreSQL & go \\
        \hline
        Zetta & nie & Basic Authentication & HTTP & Brak & javascript \\
        \hline
    \end{tabular}  
    \label{tab:koms1}
\end{table}
Zetta oraz ThingSpeak nie wspierają zarządzania urządzeniami oraz jednym protokołem do zbierania danych jest protokół http, komunikacja przy użyciu takiej technologii będzie zbyt wolna, szczególnie w momencie, gdy pentester będzie równocześnie testował wiele systemów. W odróżnieniu od tych dwóch platform Mainflux, SiteWhere oraz Kaa IoT wspierają takie protokoły, które są w stenie obsłużyć duży ruch, różnica pomiędzy nimi jest w zagwarantowaniu bezpieczeństwa. Mainflux pozwala na użycie Json Web Tokken (JWT) „tutaj 2 zdania o tym co to jest i dlaczego jest dobre”. Dodatkowo te trzy platformy używają dockerów co pozwala na szybkie uruchamianie oraz łatwą dystrybucję, atutem Mainfluxa jest język w jakim został napisany, dzięki Go rozmiar docker jest niewielki.