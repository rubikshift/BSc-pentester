\subsection[Przegląd platform IoT]{Przegląd platform IoT}
Wyrażenie Internet Rzeczy pojawiło się już kilka lat temu i od tego czasu znacznie zyskało na popularności. Już teraz wiele firm oferuje nam różnie rozwiązania od termostatu do inteligentnych urządzeń domowych takie jak lodówka, pralka, zmywarka. Żeby zapewnić niezbędne wymagania napisano wiele platform IoT. 
Jedną z najlepszych jest Google Cloud Platform pozwala między innymi na zarządzanie danymi, generuje wgląd danych za pomocą narzędzi analitycznych oraz uczenia maszynowego. Jej głównym celem jest zapewnienie inteligencji i bezpieczeństwa danych. Jest to rozwiązanie  przeznaczone głównie do dużych miast i budynków, niestety wersja darmowa ma wiele ograniczeń. 
Kolejną bardzo popularną platformą jest Cisco IoT Cloud Connect. Przeznaczona jest głównie dla urządzeń mobilnych, oferuje efektywną i wydajną łączność głosową i transmisję danych, dodatkowo zapewnia bezpieczniejszą kontrole sesji IP.
Platforma Microsoft Azure IoT jest uznawana za jedną najbardziej wydajnych platform, która ma mieć na celu gromadzenie i analizę zebranych danych. Dodatkowo oferuje szereg usług takich jak funkcję przesyłania wiadomości, mechanizmy wirtualne oraz usługi zarządzania bazami danych.
//czy jeszcze o innych komercyjnych platformach np.
\begin{itemize}
    \item Amazon Web Services lub AWS
    \item ThingWorx
    \item SAP Cloud Platform
    \item Oracle Internet of Things
    \item Bosch IoT Suite
    \item IBM Watson Internet of Things
\end{itemize}
Wszystkie wymienione powyżej platformy  IoT są platformami komercyjnymi i zbudowanie rozwiązania IoT zależy od hosta platformy i jakości wsparcia. Dlatego dobrym pomysłem jest wykorzystanie platformy Open Source. Jedną z takich platform jest DeviceHive, która oferuje stabilne połączenie z dowolnym urządzeniem za pośrednictwem interfejsu  rest api, WebSocket lub Mqtt. Dodatkowo platformę można używać zarówno z chmurą publiczną jak i prywatną. Ważnym atutem jest też, że platforma obsługuje biblioteki napisane w różnych językach.
Ciekawe rozwiązanie oferuje ThingSpeak. Jest to platforma IoT, która pozwala analizować i wizualizować dane w matlabie (bez konieczności kupowania licencji). ThingSpeak koncentruje się głównie na rejestrowaniu czujników oraz powiadomieniu użytkownika w razie potrzeby. W odróżnieniu od DeviceHive ThingSpeak oferuje przesył danych jedynie za pośrednictwem protokołu HTTP
Kolejną platformą Open Sorce na którą warto zwrócić uwagę jest mainflux, jest to nowoczesna darmowa platforma napisana w GO. Pozwala na przesyłąnie danych za pośrednictwem takich protokołów jak HTTP, MQTT, WebSocket, CoAp. Również umożliwia przygotowanie potrzebnych zasobów do połączenia takich jak rzecz oraz kanał() //provisioning
Pod względem bezpieczeństwa również wypada bardzo dobrze, ponieważ do każdej rzeczy oraz kanału generowany jest identyfikator oraz klucz który ma postać 32-znakowego ciągu heksadecymalnego.
