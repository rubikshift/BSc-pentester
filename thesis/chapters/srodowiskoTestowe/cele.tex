\subsection[Cele]{Cele}
Głównym celem środowiska testowego (dalej określane jako ST) jest ułatwienie oraz przyspieszenie pracy nad testowanym systemem. Pentester dzięki ST ma możliwość zdalnego zarządzania platformą wykonującą, która jest podpięta do testowanych urządzeń. Oznacza to, że z dowolnego miejsca może wysyłać komendy na każdą platformę wykonującą. To znacznie przyspiesza badania bezpieczeństwa, ponieważ pozwala na zautomatyzowane oraz równoczesne wykonywanie wielu testów. 
Kolejnym celem takiego środowiska jest monitorowanie aktywnych urządzeń, dzięki czemu pentester wie, że wysłana komenda zostanie obsłużona. W przypadku braku tej wiedzy czas pracy testera znacząco by wzrósł, ponieważ musiałby przy każdym wysyłaniu payloada upewniać się że platforma wykonująca otrzymała wiadomość. 
Dodatkowym atutem ST jest automatyczne odbieranie wiadomości. Testy mogą wykonywać się przez dłuższy czas. W momencie, gdy tester wysyła payload na platformę nie musi czekać na odpowiedź.  Środowisko testowe automatycznie zapisze każdą przesłaną wiadomość oraz umożliwi jej późniejsze odczytanie. To pozwala na brak ciągłej aktywności pentestera oraz zabezpiecza przed utratą wiadomości. Tester może w każdym momencie sprawdzić przesłane informacje.
Najważniejszym wymaganiem przy tworzeniu ST jest zapewnienie dobrej komunikacji. Przy testowaniu nie można pozwolić sobie na utratę wiadomości. Podczas badania bezpieczeństwa, mogą być jednocześnie wysyłane wiadomości z różnych platform wykonujących do ST jak i komendy z serwera na platformę, to oznacza, że komunikacja musi być dwustronna. Dobrą komunikację, spełniającą wszystkie wymagania można uzyskać, wykorzystując IoT.
Wyrażenie Internet Rzeczy pojawiło się już kilka lat temu i od tego czasu znacznie zyskało na popularności. Już teraz wiele firm oferuje różnie rozwiązania od termostatu do inteligentnych urządzeń domowych takie jak lodówka czy zmywarka.
Platforma IoT zapewnia łączność i normalizację przesłanych wiadomości, dzięki temu dane pobierane z różnych urządzeń, przy pomocy innych sposobów komunikacji, mają spójny format. Dodatkowym atutem jest zapewnienie bezpieczeństwa. Dane przesyłane z platformy wykonującej na środowisko testowe mogą mieć charakter danych szczególnie wrażliwych, dlatego ważnym jest żeby osoby trzecie nie miały do nich dostępu. Kolejną zaletą jest fakt, że przy IoT nie jest wymagana żadna interakcja pomiędzy człowiekiem a urządzeniem, co jest jednym z założeń przy tworzeniu środowiska testowego.
//coś dodać o autoryzacja??
