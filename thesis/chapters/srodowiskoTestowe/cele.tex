\subsection[Cele]{Cele}
Środowisko testowe (dalej określane jako ST) odpowiada za oprogramowanie serwera i obsługę komunikacji pomiędzy serwer-platforma pentestera oraz serwer-platforma wykonująca. Wymienione komponenty przedstawione są na rysunku~\ref{fig:sprzet}. % Na rysunku~\ref{fig:sprzet} przedstawiono diagram komponentów sprzętowych, .
%Środowisko testowe (dalej określane jako ST) przedstawione na rysunku~\ref{fig:sprzet} pośredniczy przy wymianie wiadomości pomiędzy platformą pentestera a platformą wykonującą. 
Głównym celem ST jest ułatwienie oraz przyspieszenie przeprowadzania testów. Pentester dzięki ST ma możliwość zdalnego zarządzania platformami wykonującymi, które są podpięte do testowanych urządzeń. Oznacza to, że z dowolnego miejsca może wysyłać komendy na każdą platformę wykonującą. To znacznie przyspiesza badania bezpieczeństwa, ponieważ pozwala na zautomatyzowane oraz równoczesne wykonywanie wielu testów. 
Kolejnym celem takiego środowiska jest monitorowanie aktywnych urządzeń, dzięki czemu pentester wie, że wysłana komenda zostanie obsłużona. W przypadku braku tej wiedzy czas pracy testera znacząco by wzrósł, ponieważ musiałby przy każdym wysyłaniu ładunku upewniać się że platforma wykonująca otrzymała wiadomość. 
Wiadomości są automatyczne odbierane. W momencie, gdy tester wysyła ładunek na platformę nie musi czekać na odpowiedź.  Środowisko testowe automatycznie zapisze każdą przesłaną wiadomość oraz umożliwi jej późniejsze odczytanie. To pozwala na brak ciągłej aktywności pentestera oraz zabezpiecza przed utratą danych. Tester może w każdym momencie sprawdzić przesłane informacje.
Najważniejszym wymaganiem przy tworzeniu ST jest zapewnienie dobrej komunikacji. To znaczy takiej, która zapewnia:
\begin{itemize}
    \item Niezawodność
    \item Bezpieczeństwo
    \item Dwukierunkowość
    \item Równoczesne przesyłanie danych do wielu urządzeń
\end{itemize}
Podczas badania bezpieczeństwa, mogą być jednocześnie wysyłane wiadomości z różnych platform wykonujących do ST jak i komendy z serwera na platformę, to oznacza, że komunikacja musi być dwustronna. Odbierane wiadomości powinny być tylko od autoryzowanych urządzeń, w przeciwnym razie wyniki mogą być przekłamane. Komunikację, spełniającą wszystkie przedstawione wymagania można uzyskać, wykorzystując IoT (Internet of Things).
Internet Rzeczy pojawił się już kilka lat temu i od tego czasu znacznie zyskał na popularności. Już teraz wiele firm oferuje różnie rozwiązania od termostatu do inteligentnych urządzeń domowych takie jak lodówka czy zmywarka.
Platforma IoT zapewnia łączność i normalizację przesłanych wiadomości, dzięki temu dane pobierane z różnych urządzeń, przy pomocy innych sposobów komunikacji, mają spójny format. Dodatkowym atutem jest zapewnienie bezpieczeństwa. Dane przesyłane z platformy wykonującej na środowisko testowe mogą mieć charakter danych szczególnie wrażliwych, dlatego ważnym jest żeby osoby trzecie nie miały do nich dostępu. Kolejną zaletą jest fakt, że przy IoT nie jest wymagana żadna interakcja pomiędzy człowiekiem a urządzeniem, co jest jednym z założeń przy tworzeniu środowiska testowego.
Wdrożenie platformy IoT jako pośrednika przy przesyłaniu wiadomości pozwoli obsłużyć wymianę informacji pomimo obecności IPv4 i translacji adresów NAT. Ponadto istniejące rozwiązania często pozwalają w prosty sposób zarządzać zarejestrowanymi urządzeniami. Platforma IoT przechowuje dane o autoryzowanych urządzeniach, tylko one mogą przesłać informację do ST. Wysłane wiadomości przez inn urządzenia zostaną odrzucone, co poprawi wiarygodność przeprowadzonych testów.
%Platforma IoT pośredniczy przy przesyłaniu wiadomości pomiędzy ST a platformę wykonującą.
%W przypadku standardowych rozwiązań zapewniających komunikację, brakuje elementu, który oferują platformy IoT. Jest to zarządzanie urządzeniami, które poprawia bezpieczeństwo.  Pentester może w każdy momencie usunąć oraz dodać zaufane urządzenie.
%//coś dodać o autoryzacja??
