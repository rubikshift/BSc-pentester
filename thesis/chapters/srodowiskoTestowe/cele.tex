\subsection[Cele]{Cele}
Głównym celem środowiska testowego jest ułatwienie oraz przyspieszenie pracy nad testowanym systemem. Pentester dzięki „środowisku ??” ma możliwość zdalnego zarządzania platformą wykonującą, która jest podpięta do testowanych urządzeń. Oznacza to, że z dowolnego miejsca może wysyłać komendy na każdą platformę wykonującą. To też znacznie przyspiesza badanie bezpieczeństwa, ponieważ pozwala na zautomatyzowane oraz równoczesne wykonywanie testów. 
Kolejnym celem takiego środowiska jest pokazanie aktywnych urządzeń, dzięki temu pentester wie, że wysłana komenda zostanie obsłużona w przypadku braku wiedzy na ten temat czas pracy testera znacząco wzrósłby, ponieważ musiałby przy każdym wysyłaniu payloada upewniać się że platforma wykonująca otrzymało wiadomość. 
Dodatkowym atutem środowiska testowego jest automatyczne odbieranie wiadomości. W momencie, gdy „platformą wykonującą podczas testowanego ataku” znalazła ważną informację może wysłać wiadomość na serwer który ją zapiszę i umożliwi późniejsze odczytanie. To pozwala na brak ciągłej aktywności pentestera.
Zapewnienie dobrej komunikacji w środowisku testowym jest najważniejszym wymaganiem. Przy testowaniu nie można pozwolić sobie na całkowitą jak i częściową utratę wiadomości. Podczas badania bezpieczeństwa, mogą być jednocześnie wysyłane wiadomości z różnych platform wykonującą na środowisko testowe jak i komendy z serwera na platforme to znaczy, że komunikacja musi być dwustronna. Dlatego do komunikacji zostanie użyte rozwiązanie IOT.  //coś dodać jeszcze
Platforma IoT zapewnia łączność i normalizację, dzięki temu dane pobierane z różnych urządzeń, przy pomocy innych sposobów komunikacji, mają spójny format danych. Dodatkowym atutem jest zapewnienie bezpieczeństwa. Dane przesyłane z platformy wykonującej na środowisko testowe mogą mieć charakter danych szczególnie wrażliwych, dlatego ważnym jest żeby osoby trzecie nie miały do nich dostępu. Kolejną zaletą jest fakt, że przy IoT nie jest wymagana żadna interakcja pomiędzy człowiekiem a urządzeniem, co jest jednym z założeń przy tworzeniu środowiska testowego.
//coś dodać o autoryzacja??
