\subsection[Mainflux]{Mainflux}
Mainflux tak jak DeviceHive obsługuje protokoły Mqtt oraz WebSocket dodatkowo obsługuje CoAP i HTTP. ThingSpeak jako że może obsłużyć jedynie HTTP przy transporcie danych to nie nadaje się do środowiska testowego głównie przez to że protokół HTTP nie jest dobry do ciągłej komunikacji...
Główną przewagą Mainfluxa nad DeviceHive jest zapewnienie lepszego bezpieczeństwa. Przy DeviceHive może zostać użyty Basic Authorization oraz Json Web Tokken (JWT) natomiast Mainflux oferuje JWT encrypted and signed tokens, OAuth2.0, public key infrastructure (PKI) and client-side certificates
//coś o plusach JWT //co to wgl jest
//jeszcze dodatkowo jak wygląda architektura mainfluxa
