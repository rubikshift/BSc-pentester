\subsection[Ogólny mechanizm działania (Michał Krakowiak)]{Ogólny mechanizm działania}
Podłączenie urządzenia wykonującego do testowanego systemu powinno skutkować rozpoznaniem go jako tzw. Human Interface Device (dalej określanego jako HID). HID to klasa urządzeń korzystających z interfejsu USB do interakcji z człowiekiem (użytkownikiem komputera). Zazwyczaj wykorzystywane są do pobierania danych wejściowych oraz prezentacji danych wyjściowych. Interfejs jest powszechnie stosowany przez producentów akcesoriów oraz dobrze udokumentowany w specyfikacji USB. Dzięki adopcji interfejsu popularne systemy takie jak Windows, Linux czy macOS są w stanie samodzielnie zidentyfikować nowe urządzenia i korzystać z nich bez potrzeby instalacji dedykowanych sterowników. Ułatwia to pracę pentestera korzystającego z realizowanego systemu, ponieważ nie ma potrzeby tworzenia i dostarczenia własnych implementacji sterownika na każdy testowany system operacyjny.
Jeszcze chciałbym coś o standardzie, adopcji, ułatwieniu w tworzeniu i instalacji urządzeń (np. to, że system często jest w stanie sam rozpoznać i skorzystać z urządzenia bez sterownika)
