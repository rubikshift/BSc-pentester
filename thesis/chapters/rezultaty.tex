\section[Rezultaty]{Rezultaty}
\label{chapter:results}
Utworzony system ma na celu badanie bezpieczeństwa, natomiast rezultatem jest możliwość przeprowadzenia testów. Testy pokazują występowanie podatności. 
Oceny ryzyka będzie wzorowana na model DREAD (ang. Damage, Reproducibility, Exploitability, Affected users, Discoverability). Standardowy system oceniania polega na przydzieleniu każdej z pięciu kategorii wartość od 1 do 10. Im wyższa pozycja, tym ryzyko jest większe. Zastosowano 3 stopniową skalę (niski, średni oraz wysoki poziom zagrożenia). Oceny przyznawano subiektywnie.
\begin{itemize}
    \item Szkody (ang. Damage) oznacza, jak duże mogą być konsekwencje wykorzystania podatności
    \item Odtwarzalność (ang. Reproducibility) oznacza, jak łatwo można wykorzystać podatność
    \item Możliwość wykorzystania (ang. Exploitability), mówi o tym, ile czasu jest potrzebne do uruchomienia potencjalnego ataku
    \item Dotknięci użytkownicy (ang. Affected users), ta kategoria ocenia podatność pod względem liczby poszkodowanych osób
    \item Wykrywalność (ang. Discoverability), ocena pod względem wykrywalności zagrożenia, to znaczy jak łatwo użytkownik zorientuje się, że jest (lub był) atakowany
\end{itemize}

\subsection[Podatności na podszywanie się pod HID]{Podatności na podszywanie się pod HID}
\subsubsection[Wykazanie podatności]{Wykazanie podatności}
Scenariusz~\ref{sce:klawiatura} opisuje test polegający na zdalnym wprowadzeniu dowolnej sekwencji klawiszy w systemie komputerowym. Zostały w nim zawarte niezbędne informacje do przeprowadzenia testu podatności. Scenariusz opisuje urządzenia oraz założenia początkowe wymagane do poprawnego wykonania testu. Najważniejszym urządzeniem jest platforma wykonującą (tzn. rpi identyfikujący się jako klawiatura), którą pentester podłącza do komputera z systemem \textit{Windows}. Po podłączeniu pentester przygotowuje payload. Przykład na listinigu~\ref{lst:hid}.
\begin{lstlisting}[language={},label={lst:hid},caption={Przykładowy payload}]
{
    'deviceType': 'KEYBOARD',
    'name': 'test',
    'idVendor': '0x1d6b',
    'idProduct': '0x0104',
    'serial': 'fedcba9876543210',
    'manufacturer': 'BSC',
    'product': 'BSC',
    'payloads':[{'delay':0.5,'key':'r','leftGUI':'True'},
                {'delay':0.5,'text':'cmd.exe\\n'}]
}
\end{lstlisting}
Pierwsze elementy odpowiadają za konfigurację urządzenia, z kolei ostatni \textit{payloads} zawiera listę komend (tzn. sekwencje klawiszy które mają zostać wysłane). W przedstawionym ładunku pierwsza komenda ma zadanie uruchomić okno umożliwiające włączenie dowolnego programu. Skrót odpowiadający za uruchomienie tego okna w systemie \textit{Windows} to \textit{win + r}, gdzie klawiszowi \textit{win} odpowiada \textit{leftGUI}. W kolejnym argumencie został podany tekst \textit{cmd.exe}, który zostanie podany na wejście. Wynikiem obsłużenia przedstawionego ładunku, będzie otworzenie konsoli, następnie można podać kolejne argumenty, które wyślą dowolną inną kombinację klawiszy. Pentester, może ustawić opóźnienie, wykonania komendy.
Pentester może w każdym momencie wysłać nowy ładunek na aktywne urządzenie wykonujące za pośrednictwem interfejsu użytkownika. Przykładowy payload z listingu~\ref{lst:hid}, pozwala na uruchomienie konsoli. Dalej można zlecić kolejne instrukcje, np. skrypt powłoki, uzyskując podatność typu RCE. 
\subsubsection[Ocena Ryzyka]{Ocena Ryzyka}
\begin{itemize}
    \item Szkody - duże, szkody spowodowane przez zdalne wykonywanie kodu mogą być bardzo duże i nie są zależne od uprawnień zalogowanego użytkownika. Pomimo braku pełnych praw administratora pentester może eskalować uprawnienia, wykorzystując inne luki. Podatność RCE daje narzędzia do przejmowania danych oraz wykonywania skryptów.
    \item Odtwarzalność - średnia, nie jest łatwa ze względu na to, że pentester musi mieć kontakt fizyczny z testowanym komputerem lub nieświadomy użytkownik podłączy urządzenie. Przez fakt, że następuje postęp miniaturyzacja elektroniki oraz wiele osób jest nieświadomych zagrożeń wynikających z wpinania niesprawdzonego sprzętu ocena odtwarzalności klasyfikuje się na średnim poziomie. Szerszy opis znajduje się w rozdziale~\ref{bez:int}. %Natomiast w momencie, gdy platforma wykonujące jest już podłączona odtwarzalność jest już dużo prostsza ponieważ pentester może w dowolnej chwili wykonać test.
    \item Możliwość wykorzystania - średnia, uruchomienie testu zajmuje kilkadziesiąt sekund z czego większość jest poświęcona na konfigurację wstępną platformy wykonującej. %Szerzej zostało opisane w rozdziale 
    \item Dotknięci użytkownicy - średnia, potencjalnym celem może być osoba aktualnie zalogowana lub inne osoby korzystający  z komputera, w przypadku gdy zalogowany użytkownik ma uprawnienia administratora lub pentester wykorzysta inne luki do zdobycia tych uprawnień.
    \item Wykrywalność - mała, poziom jest dosyć niski, gdyż opóźnienie przy wysyłaniu kolejnych sekwencji znaków jest minimalne i nieświadomy użytkownik może się nie zorientować, że właśnie został wykonany kod. Po podłączeniu i uruchomieniu urządzenia system \textit{Windows} informuje o nierozpoznanym urządzeniu, lecz większość nieświadomych użytkowników może zignorować komunikat.
\end{itemize}
\subsection[Podatność na zatruwanie DNS (Jakub Wyka)]{Podatność na zatruwanie DNS}
\subsubsection[Wykazanie podatności]{Wykazanie podatności}
W rozdziale~\ref{chapter:usbethernet} omawiana była realizacja symulowania działania karty sieciowej.
W ramach tej funkcjonalności powstał scenariusz (rozdział~\ref{sce:phar}) 
zakładający zmianę rekordu DNS. W celu przeprowadzenia testu należy przesłać odpowiednio przygotowaną 
wiadomość do urządzenia wykonującego. 

\begin{lstlisting}[language={},label={lst:ph},caption={Przykładowa wiadomość}]
    {
        {
            'deviceType': 'ETHER',
            'name': 'test',
            'idVendor': '0x1d6b',
            'idProduct': '0x0104',
            'serial': 'fedcba9876543210',
            'manufacturer': 'BSC',
            'product': 'BSC',
            'payloads':
            [
                {
                    'scenario': 'SPOOFING',
                    'params': 
                    [
                        {
                            'ipAddress': '192.168.1.1',
                            'hostname': 'wp.pl',
                            'duration':'240'
                        }
                    ]
                }
            ]
        }
    }
\end{lstlisting}
Jak już wspomniano w rozdziale ~\ref{chapter:mitmMK} przykładowa konfiguracja z listingu~\ref{lst:ph} 
zawiera podstawowe parametry konfiguracyjne dla tworzonego gadżetu USB. Należy również podać 
nazwę realizowanego scenariusza. W tym przypadku jest to \textit{SPOOFING}.
Za pomocą pola \textit{params} przekazywane są parametry specyficzne 
dla wybranego wcześniej scenariusza. 
Jest to \textit{ipAddress}  który określa adres IP w dodawanym rekordzie DNS, a nazwę 
hosta definiuje pole \textit{hostname}. \textit{Duration} jest parametrem określającym czas 
trwania testu. Powyższe parametry wystarczają aby dodać lokalny wpis DNS na urządzeniu wykonującym 
i przekierować użytkownika systemu do złośliwej witryny. Jeśli podany adres IP odpowiada adresowi 
przypisanemu do interfejsu wlan0 urządzenia wykonującego, użytkownik zostaje przekierowany do dokumentu 
hostowanego przez serwer http na urządzeniu wykonującym. Za pomocą skryptu php wpisane przez niego 
wrażliwe dane zostają zapisane na dysku, a następnie zostają odesłane w raporcie testu.

\subsubsection[Ocena Ryzyka]{Ocena Ryzyka}
\begin{itemize}
    \item Szkody - duże, przekierowanie użytkownika na stronę podszywającą się bez jego 
    wiedzy może przynieść poważne konswekwencje. W szczególności jest to kradzież wrażliwych 
    danych, które mogą zostać wykorzystane m.in. do kradzieży konta na portalu lub dostęp do 
    konta bankowego. Oznaczać to może stratę faktycznych dóbr materialnych przez ofiarę.
    \item Odtwarzalność - średnia, opisana w rozdziale ~\ref{chapter:ormitm}.
    \item Możliwość wykorzystania - średnia, opisana w rozdziale ~\ref{chapter:ormitm}.
    \item Ilość dotkniętych użytkowników - mała, zmiana rekordu DNS dotyczy tylko użytkownika, 
    do którego stacji roboczej podłączone jest urządzenie wykonujące.
    \item Wykrywalność - średnia, opisana w rozdziałach ~\ref{subsec:wykrywalnoscJW} i ~\ref{subsec:wykrpharming}. 
    Użytkownik jest w stanie wykryć symulację ataku, jednak wymaga to od niego podstawowej wiedzy na temat 
    certyfikatów. Ważne jest też zwracanie uwagi na komunikaty systemowe.
\end{itemize}

\subsection[Podatność na przechwycenie ruchu sieciowego (Michał Krakowiak)]{Podatność na przechwycenie ruchu sieciowego}
\subsubsection[Wykazanie podatności]{Wykazanie podatności}
\label{chapter:mitmMK}
W rozdziale~\ref{chapter:usbethernet} omawiana była realizacja symulowania działania karty sieciowej.
W ramach tej funkcjonalności powstał scenariusz (rozdział~\ref{sce:mitm}) zakładający przechwycenie ruchu sieciowego.
W celu przeprowadzenia testu system należy odpowiednio uzbroić, przekazując wymaganą konfigurację.
\begin{lstlisting}[language={},label={lst:mitm},caption={Przykładowa konfiguracja}]
{
    'deviceType': 'ETHER',
    'name': 'test',
    'idVendor': '0x1d6b',
    'idProduct': '0x0104',
    'serial': 'fedcba9876543210',
    'manufacturer': 'BSC',
    'product': 'BSC',
    'nfqueue': True,
    'payloads': [
        {
            'scenario': 'INJECT',
            'code': '<script>alert(\"Hello World\");</script>',
            'count': 3
        }
    ]
}
\end{lstlisting}
Przykładowa konfiguracja z listingu~\ref{lst:mitm} zawiera podstawowe parametry konfiguracyjne dla tworzonego gadżetu USB.
Pojawia się pole \textit{nfqueue} (skrócona nazwa pochodząca od \textit{netfilter queue}), które jest odpowiedzialne za poinformowanie systemu o konieczności przekierowania ruchu sieciowego do dedykowanej kolejki.
\textit{Netfilter queue} pozwala na filtrowanie i modyfikowanie przekierowanych do niej danych z poziomu przestrzeni adresowej użytkownika.
W związku z implementacją więcej niż jednego scenariusza z wykorzystaniem symulowanej karty sieciowej wymagane jest wskazanie oczekiwanej funkcjonalności (pole \textit{scenario}).
Test wykorzystuje fakt, że po przyłączeniu nowego interfejsu system \textit{Windows~10} decydował o kierowaniu do niego wszystkie dane.
System nie zweryfikował autentyczności nowego interfejsu.
Takie zachowanie umożliwiło wykorzystanie pośredniczenia w połączeniu.
Dedykowany system był wstanie przechwycić każdą przesłaną informację.
Na potrzeby analizy krytyczności zagrożenia wprowadzono obsługę modyfikacji nagłówków HTTP:
\begin{itemize}
    \item podmianę wersji protokołu na 1.0
    \item ustawienie preferowanej zawartości danych jako zwykły tekst
    \item aktualizacja rozmiaru danych HTTP.
\end{itemize}
Pozwoliło to wstrzyknąć kod i wykonać przesłany w ładunku.
Daje to możliwość dalszej eskalacji i wykorzystania innych podatności.
Warunkiem pomyślnej exploitacji jest obecność niezaszyfrowanego ruchu, od którego już się odchodzi.
\subsubsection[Ocena ryzyka]{Ocena ryzyka}
\label{chapter:ormitm}
Na podstawie przeprowadzonego testu stwierdzono średnie ryzyko zagrożenia. Szczegółowa ocena:
\begin{itemize}
    \item Szkody - duże, obecność zdalnie sterowanego urządzenia w połączeniu internetowym daje takie możliwości przechwytywanie plików cookie lub wstrzyknięcie kodu javascript
    \item Odtwarzalność - średnia, odtworzenie scenariusza wymaga dostępu do fizycznego interfejsu komputera, jednak należy uwzględniać, że koszt miniaturowej elektroniki maleje; złośliwe urządzenia mogą być dystrybuowane z wykorzystaniem metod socjotechnicznych (np. w postaci darmowych gadżetów reklamowych)  
    \item Możliwość wykorzystania - średnia, skuteczne wykorzystanie wymaga nie tylko odpowiedniego oprogramowania, ale też dystrybucji fizycznego urządzenia
    \item Ilość dotkniętych użytkowników - mała, skuteczne wykorzystanie przejęcia ruchu sieciowego zagraża przede wszystkim użytkownikom danego systemu komputerowego
    \item Wykrywalność - średnia, szerzej opisana w rozdziałach~\ref{subsec:wykrywalnoscMK} i~\ref{subsec:wykrywalnoscJW}, użytkownik ma szansę wykryć zagrożenie, jednak zignorowanie komunikatu w przypadku nieświadomych użytkowników może skutkować utrudnieniami w późniejszym wykryciu.
\end{itemize}