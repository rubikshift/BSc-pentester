\section[Rezultaty]{Rezultaty}
\label{chapter:results}

\subsection[Rpi jako klawiatura]{Rpi jako klawiatura}
\subsubsection[Stan początkowy]{Stan początkowy}
Do sprawdzenia podatności można wykorzystać platforme wykonującą (tzn. rpi identyfikujący się jako klawiatura).
Opis jak rpi może przedstwiać się jako klawiatura oraz wsyłać sekwencje znaków znajduję się w rozdziale o symulowaniu działania klawiatury USB.
System, jaki był testowany to windows 10. Do przeprowadzenia testu potrzebny jest jeszcze zwykły kabel USB, który będzie służył do połączenia rpi z systemem.
\subsubsection[Dane Wejściowe]{Opis działania} %Dane testowe
Pentester podłącza platforme wykonującą do komputeraz z systemem windows. Następnie przygotowuje payload, przykładowy ładunek może wyglądać następująco:
\begin{lstlisting}[language={},caption={Przykładowy payload}]
{'deviceType': 'KEYBOARD','name': 'test','idVendor': '0x1d6b','idProduct': '0x0104','serial': 'fedcba9876543210','manufacturer': 'BSC','product': 'BSC','payloads':[{'delay':0.5,'key':'r','leftGUI':'True'},{'delay':0.5,'text':'cmd.exe\\n'}]}. 
\end{lstlisting}
Pierwsze elementy odpowiadają za konfigurację urządzenia, z kolei ostatni 'payloads' zawiera listę komend (tzn. sekwencje klawiszy które mają zostać wysłane). W przedstawionym ładunku pierwsza komenda ma zadanie uruchomić okno umożliwiające włączenie dowolnego programu. W kolejnym argumencie został podany 'cmd.exe \ n' jest to tekst, który zostanie podany na wejście. Przedstawiony payload owtiera konsole, następnie można podać kolejne argumenty, które wyślą dowolną kombinację klawiszy. Pentester, może ustawić opóźnienie z jakim rpi ma wysłać wiadomości.
\subsubsection[Wykazanie Podatności]{Wykazanie podatności}
Pentester może w każdym momencie wysłać ładunek na aktywne rpi za pośrednictwem dashboardu. Przygytowany payload taki jak w powyższym opisie pozwoli na uruchomienie konsoli, po czym może wywołać dowolną instrukcję. Opóźnienie przy wysyłaniu znaków jest minimalne i przeciętny użytkownik może się nie zorientować, że właśnie został wykonany kod.
\subsubsection[Ocena Podatności]{Ocena podatności}
Do oceny wskazanej podatność zostanie wykorzystany model DREAD (ang. Damage, Reproducibility, Exploitability, Affected users, Discoverability).
\begin{itemize}
    \item Damage oznacza, jak duże mogą być konsekwencje wykorzystania podatności. tutaj opis
    \item Reproducibility oznacza, jak łatwo można wykorzystać podatność. tutaj opis
    \item Exploitability, mówi o tym, ile czasu jest potrzebne do uruchomienia ataku. tutaj opis
    \item Affected users, ta kategoria ocenia podatność pod względem liczby poszkodowanych osób. tutaj opis
    \item Discoverability, ocena pod względem wykrywalności zagrożenia, to znaczy jak łatwo użytkownik zorientuje się, że jest (lub był) atakowany. tutaj opis
\end{itemize}


\subsection[Rpi jako klawiatura]{Rpi jako karta sieciowa (atak typu Mitm)}
\subsubsection[Stan początkowy]{Stan początkowy}

\subsubsection[Dane Wejściowe]{Opis działania} %Dane testowe
Pentester podłącza platforme wykonującą do komputeraz z systemem windows. Następnie przygotowuje payload, przykładowy ładunek może wyglądać następująco:
\subsubsection[Wykazanie Podatności]{Wykazanie podatności}
\subsubsection[Ocena Podatności]{Ocena podatności}
Do oceny wskazanej podatność zostanie wykorzystany model DREAD (ang. Damage, Reproducibility, Exploitability, Affected users, Discoverability).
\begin{itemize}
    \item Damage oznacza, jak duże mogą być konsekwencje wykorzystania podatności. tutaj opis
    \item Reproducibility oznacza, jak łatwo można wykorzystać podatność. tutaj opis
    \item Exploitability, mówi o tym, ile czasu jest potrzebne do uruchomienia ataku. tutaj opis
    \item Affected users, ta kategoria ocenia podatność pod względem liczby poszkodowanych osób. tutaj opis
    \item Discoverability, ocena pod względem wykrywalności zagrożenia, to znaczy jak łatwo użytkownik zorientuje się, że jest (lub był) atakowany. tutaj opis
\end{itemize}

\subsection[Rpi jako klawiatura]{Rpi jako karta sieciowa (atak typu Pharming)}
\subsubsection[Stan początkowy]{Stan początkowy}

\subsubsection[Dane Wejściowe]{Opis działania} %Dane testowe
Pentester podłącza platforme wykonującą do komputeraz z systemem windows. Następnie przygotowuje payload, przykładowy ładunek może wyglądać następująco:
...
\subsubsection[Wykazanie Podatności]{Wykazanie podatności}
\subsubsection[Ocena Podatności]{Ocena podatności}
Do oceny wskazanej podatność zostanie wykorzystany model DREAD (ang. Damage, Reproducibility, Exploitability, Affected users, Discoverability).
\begin{itemize}
    \item Damage oznacza, jak duże mogą być konsekwencje wykorzystania podatności. tutaj opis
    \item Reproducibility oznacza, jak łatwo można wykorzystać podatność. tutaj opis
    \item Exploitability, mówi o tym, ile czasu jest potrzebne do uruchomienia ataku. tutaj opis
    \item Affected users, ta kategoria ocenia podatność pod względem liczby poszkodowanych osób. tutaj opis
    \item Discoverability, ocena pod względem wykrywalności zagrożenia, to znaczy jak łatwo użytkownik zorientuje się, że jest (lub był) atakowany. tutaj opis
\end{itemize}