\section[Rezultaty]{Rezultaty}
\label{chapter:results}
\subsection[Wstęp]{Wstęp}
Utworzony system ma na celu badanie bezpieczeństwa, natomiast rezultatem jest możliwość przeprowadzenia testów. Testy pokazują występowanie podatności. 
Do oceny ryzyka wskazanej podatność zostanie wykorzystany model DREAD (ang. Damage, Reproducibility, Exploitability, Affected users, Discoverability). System oceniania polega na przydzieleniu każdej z pięciu kategorii wartość od 1 do 10. Im wyższa ocena, tym ryzyko jest większe.  
\begin{itemize}
    \item Obrażenia (ang. Damage) oznacza, jak duże mogą być konsekwencje wykorzystania podatności
    \item Odtwarzalność (ang. Reproducibility) oznacza, jak łatwo można wykorzystać podatność
    \item Możliwość wykorzystania (ang. Exploitability), mówi o tym, ile czasu jest potrzebne do uruchomienia potencjalnego ataku
    \item Dotknięci użytkownicy (ang. Affected users), ta kategoria ocenia podatność pod względem liczby poszkodowanych osób
    \item Wykrywalność (ang. Discoverability), ocena pod względem wykrywalności zagrożenia, to znaczy jak łatwo użytkownik zorientuje się, że jest (lub był) atakowany
\end{itemize}
\subsection[Podatność remote code execution]{Test podatności na podszycie sie pod hid}
% \subsubsection[Podatność remote code execution]{Stan początkowty}
% Do sprawdzenia podatności wykorzystana zostanie platforma wykonującą (tzn. rpi identyfikujący się jako klawiatura).
% Opis jak rpi może przedstwiać się jako klawiatura oraz wsyłać sekwencje znaków znajduję się w rozdziale o symulowaniu działania klawiatury USB.
% System, jaki był testowany to windows 10. Do przeprowadzenia testu potrzebny jest jeszcze kabel USB, który będzie służył do połączenia rpi z systemem.
\subsubsection[Dane Wejściowe]{Testowanie podatności} %Wykazanie podatności
Scenariusz~\ref{sce:klawiatura} opisuje test polegający na zdalnym wprowadzeniu dowolnej sekwencji klawiszy w systemie komputerowym. Zostały w nim zawarte niezbędne informacje do przeprowadzenia testu podatności. Scenariusz opisuje urządzenia oraz założenia początkowe wymagane do poprawnego wykonania testu. Najważniejszym urządzeniem jest platforma wykonującą (tzn. rpi identyfikujący się jako klawiatura), którą pentester podłącza do komputera z systemem windows. Po podłączeniu pentester przygotowuje payload, przykładowy ładunek może wyglądać następująco:
\begin{lstlisting}[language={},caption={Przykładowy payload}]
{
    'deviceType': 'KEYBOARD',
    'name': 'test',
    'idVendor': '0x1d6b',
    'idProduct': '0x0104',
    'serial': 'fedcba9876543210',
    'manufacturer': 'BSC',
    'product': 'BSC',
    'payloads':[{'delay':0.5,'key':'r','leftGUI':'True'},
                {'delay':0.5,'text':'cmd.exe\\n'}]
}
\end{lstlisting}
Pierwsze elementy odpowiadają za konfigurację urządzenia, z kolei ostatni 'payloads' zawiera listę komend (tzn. sekwencje klawiszy które mają zostać wysłane). W przedstawionym ładunku pierwsza komenda ma zadanie uruchomić okno umożliwiające włączenie dowolnego programu. W kolejnym argumencie został podany 'cmd.exe' jest to tekst, który zostanie podany na wejście. Wynikiem obsłużenia przedstawionego ładunku, będzie otworzenie konsoli, następnie można podać kolejne argumenty, które wyślą dowolną kombinację klawiszy. Pentester, może ustawić opóźnienie, które odpowiada za czekanie na wykonanie komendy.
Pentester może w każdym momencie wysłać ładunek na aktywne rpi za pośrednictwem interfejsu użytkownika. Przygotowany payload taki jak w powyższym opisie pozwoli na uruchomienie konsoli, po czym może wywołać dowolną instrukcję. 
\subsubsection[Ocena Podatności]{Ocena Ryzyka}
\begin{itemize}
    \item Obrażenia spowodowane przez remote code execution mogą być bardzo duże i zależą od uprawnień zalogowanego użytkownika. Nawet w przypadku gdy użytkownik nie jest zarejestrowany jako administrator podatność RCE daje narzędzia do przejmowania danych oraz wykonywania niektórych skryptów.
    \item Odtwarzalność testu jest dosyć trudna ze względu na to, że pentester musi mieć kontakt fizyczny z testowanym komputerem lub nieświadomy  użytkownik podłączy urządzenie. Natomiast w momencie, gdy platforma wykonujące jest już podłączona odtwarzalność jest już dużo prostsza ponieważ pentester może w dowolnej chwili wykonać test.
    \item Możliwość wykorzystania, uruchomienie testu zajmuje kilkadziesiąt sekund z czego większość jest poświęcona na konfigurację wstępną platformy wykonującej.
    \item Dotkniętym użytkownikiem może być osoba aktualnie zalogowana lub inne osoby korzystający  z komputera w przypadku gdy zalogowany użytkownik ma uprawnienia administratora.
    \item Wykrywalność testu jest dosyć mała, gdyż opóźnienie przy wysyłaniu kolejnych sekwencji znaków jest minimalne i nieświadomy użytkownik może się nie zorientować, że właśnie został wykonany kod. Po podłączeniu i uruchomieniu urządzenia system Windows informuje o nierozpoznanym urządzeniu, to może spowodować ....
\end{itemize}