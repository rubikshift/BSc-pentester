\section[Rezultaty]{Rezultaty}
o czym pisać:
\begin{itemize}
    \item przygotowanie omówienie "formalnej" struktury "rezultatu" (na przykładzie sprawozdań z testu penetracyjnego lub zgłoszeń do bug bounty)
    \item przedstawienie przebiegu (przygotowania i technicznej realizacji) dla każdego zrealizowanego scenariusza
    \item wskazanie znalezionych znalezionych podatności (np. RCE - zdalne wykonanie kodu) i określenie zagrożenia jakie stanowią (np. trzeba uwzględnić czynnik, że napastnik musi mieć dostęp do portu USB)
\end{itemize}

%o czym pisać:
%\begin{itemize}
%    \item Podsumowanie wykonanych prac (tzn. jaki system testowaliśmy (windows 10), ile czasu na tym spędziliśmy w osobogodzinach, na jakim koncie windowsa pracowaliśmy/testowaliśmy (testy bez posiadania uprawnień administracyjnych.) oraz na czym skupialiśmy się przy testowaniu (przejęcie kontroli nad systemem, wykradnięcie danych))
%    \item Wnioski z testów (tzn. wykrycie podatności (zidentyfikowanie kilka sposobów na przejęcie kontroli), oraz opisanie wykorzystania tej podatności)
%    \item Zalecenia (co zrobić żeby uniknąć tych zagrożeń (może edukacja ludzi, żeby nie wpinali każdego urządzenia))
%    \item Podsumowanie techniczne (tzn. opisać krótko jak wykorzystać podatności (wpiąć urządzenie pod usb i zarejestrować się np. jako klawiatura))
%    \item wykres z liczbą wykrytych zagrożeń (podzielony na poziom)
%    \item Znalezione podatności tablea (z polami: poziom zagrożenia np. krytyczne, nazwa podatności, opis podatności)
%\end{itemize}
%
%przykład tabeli
%
%\begin{table}[H]
%    \begin{tabular}{|p{3cm}|p{4cm}|p{7cm}|}
%        \hline
%        \textbf{poziom zagrżenia} & \textbf{nazwa podatności}  & \textbf{opis podatności} \\
%        \hline
%        Krytyczne & Wykonanie dowolnego payloada & (coś w stylu) Windows pozwala na dowolne wykonanie payloada (urządzenie które rejestruje się jako klawiatura może wpisywać dowolne znaki przy dowolnej prędkości )\\
%        \hline
%    \end{tabular}
%    \caption{Tabela podatności }  
%    \label{tab:podatnosc1}
%\end{table}	
	
	
	
    