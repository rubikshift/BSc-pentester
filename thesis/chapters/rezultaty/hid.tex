\subsection[Podatności na podszywanie się pod HID (Kacper Połom)]{Podatności na podszywanie się pod HID}
\subsubsection[Wykazanie podatności]{Wykazanie podatności}
Scenariusz~\ref{sce:klawiatura} opisuje test polegający na zdalnym wprowadzeniu dowolnej sekwencji klawiszy w systemie komputerowym. Zostały w nim zawarte niezbędne informacje do przeprowadzenia testu podatności. Scenariusz opisuje urządzenia oraz założenia początkowe wymagane do poprawnego wykonania testu. Najważniejszym urządzeniem jest platforma wykonującą (tzn. rpi identyfikujący się jako klawiatura), którą pentester podłącza do komputera z systemem windows. Po podłączeniu pentester przygotowuje payload, przykładowy ładunek może wyglądać następująco:
\begin{lstlisting}[language={},caption={Przykładowy payload}]
{
    'deviceType': 'KEYBOARD',
    'name': 'test',
    'idVendor': '0x1d6b',
    'idProduct': '0x0104',
    'serial': 'fedcba9876543210',
    'manufacturer': 'BSC',
    'product': 'BSC',
    'payloads':[{'delay':0.5,'key':'r','leftGUI':'True'},
                {'delay':0.5,'text':'cmd.exe\\n'}]
}
\end{lstlisting}
Pierwsze elementy odpowiadają za konfigurację urządzenia, z kolei ostatni 'payloads' zawiera listę komend (tzn. sekwencje klawiszy które mają zostać wysłane). W przedstawionym ładunku pierwsza komenda ma zadanie uruchomić okno umożliwiające włączenie dowolnego programu. Skrót odpowiadający z uruchomienie tego okna w systemie Windows to win + r, gdzie klawiszowi win odpowiada leftGUI. W kolejnym argumencie został podany 'cmd.exe' jest to tekst, który zostanie podany na wejście. Wynikiem obsłużenia przedstawionego ładunku, będzie otworzenie konsoli, następnie można podać kolejne argumenty, które wyślą dowolną kombinację klawiszy. Pentester, może ustawić opóźnienie, które odpowiada za czekanie na wykonanie komendy.
Pentester może w każdym momencie wysłać ładunek na aktywne rpi za pośrednictwem interfejsu użytkownika. Przygotowany payload taki jak w powyższym opisie pozwoli na uruchomienie konsoli, po czym może wywołać dowolną instrukcję. 
\subsubsection[Ocena Ryzyka]{Ocena Ryzyka}
\begin{itemize}
    \item Obrażenia - (duże), szkody spowodowane przez zdalne wykonywanie kodu mogą być bardzo duże i nie są zależne od uprawnień zalogowanego użytkownika. Pomimo braku pełnych praw administratora pentester może eskalować uprawnienia, wykorzystując inne luki. Podatność RCE daje narzędzia do przejmowania danych oraz wykonywania skryptów.
    \item Odtwarzalność - (średnia) nie jest łatwa ze względu na to, że pentester musi mieć kontakt fizyczny z testowanym komputerem lub nieświadomy użytkownik podłączy urządzenie. Przez fakt, że następuje postęp miniaturyzacja elektroniki oraz wiele osób jest nieświadomych zagrożeń wynikających z wpinania niesprawdzonego sprzętu ocena odtwarzalności klasyfikuje się na średnim poziomie.Szerszy opis znajduje się w rozdziale~\ref{bez:int}. %Natomiast w momencie, gdy platforma wykonujące jest już podłączona odtwarzalność jest już dużo prostsza ponieważ pentester może w dowolnej chwili wykonać test.
    \item Możliwość wykorzystania - (średnia), uruchomienie testu zajmuje kilkadziesiąt sekund z czego większość jest poświęcona na konfigurację wstępną platformy wykonującej. %Szerzej zostało opisane w rozdziale 
    \item Dotknięci użytkownicy - (średnia) potencjalnym celem może być osoba aktualnie zalogowana lub inne osoby korzystający  z komputera, w przypadku gdy zalogowany użytkownik ma uprawnienia administratora lub pentester wykorzysta inne luki do zdobycia tych uprawnień.
    \item Wykrywalność - (mała) poziom jest dosyć niski, gdyż opóźnienie przy wysyłaniu kolejnych sekwencji znaków jest minimalne i nieświadomy użytkownik może się nie zorientować, że właśnie został wykonany kod. Po podłączeniu i uruchomieniu urządzenia system Windows informuje o nierozpoznanym urządzeniu, lecz większość nieświadomych użytkowników ignoruje komunikat.
\end{itemize}