\subsection[Podatność na przechwycenie ruchu sieciowego (Michał Krakowiak)]{Podatność na przechwycenie ruchu sieciowego}
\subsubsection[Wykazanie podatności]{Wykazanie podatności}
\label{chapter:mitmMK}
W rozdziale~\ref{chapter:usbethernet} omawiana była realizacja symulowania działania karty sieciowej.
W ramach tej funkcjonalności powstał scenariusz (rozdział~\ref{sce:mitm}) zakładający przechwycenie ruchu sieciowego.
W celu przeprowadzenia testu system należy odpowiednio uzbroić, przekazując wymaganą konfigurację.
\begin{lstlisting}[language={},label={lst:mitm},caption={Przykładowa konfiguracja}]
{
    'deviceType': 'ETHER',
    'name': 'test',
    'idVendor': '0x1d6b',
    'idProduct': '0x0104',
    'serial': 'fedcba9876543210',
    'manufacturer': 'BSC',
    'product': 'BSC',
    'nfqueue': True,
    'payloads': [
        {
            'scenario': 'INJECT',
            'code': '<script>alert(\"Hello World\");</script>',
            'count': 3
        }
    ]
}
\end{lstlisting}
Przykładowa konfiguracja z listingu~\ref{lst:mitm} zawiera podstawowe parametry konfiguracyjne dla tworzonego gadżetu USB.
Pojawia się pole \textit{nfqueue} (skrócona nazwa pochodząca od \textit{netfilter queue}), które jest odpowiedzialne za poinformowanie systemu o konieczności przekierowania ruchu sieciowego do dedykowanej kolejki.
\textit{Netfilter queue} pozwala na filtrowanie i modyfikowanie przekierowanych do niej danych z poziomu przestrzeni adresowej użytkownika.
W związku z implementacją więcej niż jednego scenariusza z wykorzystaniem symulowanej karty sieciowej wymagane jest wskazanie oczekiwanej funkcjonalności (pole \textit{scenario}).
Test wykorzystuje fakt, że po przyłączeniu nowego interfejsu system \textit{Windows~10} decydował o kierowaniu do niego wszystkie dane.
System nie zweryfikował autentyczności nowego interfejsu.
Takie zachowanie umożliwiło wykorzystanie pośredniczenia w połączeniu.
\textit{BSc-pentestera} był wstanie przechwycić każdą przesłaną informację.
Na potrzeby analizy krytyczności zagrożenia wprowadzono obsługę modyfikacji nagłówków HTTP:
\begin{itemize}
    \item podmianę wersji protokołu na 1.0,
    \item ustawienie preferowanej zawartości danych jako zwykły tekst,
    \item aktualizacja rozmiaru danych HTTP.
\end{itemize}
Pozwoliło to wstrzyknąć kod i wykonać przesłany w ładunku.
Daje to możliwość dalszej eskalacji i wykorzystania innych podatności.
Warunkiem pomyślnej exploitacji jest obecność niezaszyfrowanego ruchu, od którego już się odchodzi.
\subsubsection[Ocena ryzyka]{Ocena ryzyka}
\label{chapter:ormitm}
Na podstawie przeprowadzonego testu stwierdzono średnie ryzyko zagrożenia. Szczegółowa ocena:
\begin{itemize}
    \item Szkody - duże, obecność zdalnie sterowanego urządzenia w połączeniu internetowym daje takie możliwości jak np. przechwytywanie plików cookie lub wstrzyknięcie kodu javascript
    \item Odtwarzalność - średnia, odtworzenie scenariusza wymaga dostępu do fizycznego interfejsu komputera, jednak należy uwzględniać, że koszt miniaturowej elektroniki maleje; złośliwe urządzenia mogą być dystrybuowane z wykorzystaniem metod socjotechnicznych (np. w postaci darmowych gadżetów reklamowych)  
    \item Możliwość wykorzystania - średnia, skuteczne wykorzystanie wymaga nie tylko odpowiedniego oprogramowania, ale też dystrybucji fizycznego urządzenia
    \item Ilość dotkniętych użytkowników - mała, skuteczne wykorzystanie przejęcia ruchu sieciowego zagraża przede wszystkim użytkownikom danego systemu komputerowego
    \item Wykrywalność - średnia, szerzej opisana w rozdziałach~\ref{subsec:wykrywalnoscMK} i~\ref{subsec:wykrywalnoscJW}, użytkownik ma szansę wykryć zagrożenie, jednak zignorowanie komunikatu w przypadku nieświadomych użytkowników może skutkować utrudnieniami w późniejszym wykryciu.
\end{itemize}