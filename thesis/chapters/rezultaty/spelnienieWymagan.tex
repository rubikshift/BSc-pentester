\subsection[Podsumowanie rezultatów implementacyjnych (Jakub Wyka)]{Podsumowanie rezultatów implementacyjnych}
\subsubsection[Spełnienie wymagań funkcjonalnych]{Spełnienie wymagań funkcjonalnych}
Zaimplementowany system pozwala na wykonanie wszystkich akcji wyszczególnionych w~rozdziale~\ref{subsec:fun}. 
Wszystkie, manualnie przeprowadzane, testy systemowe zakończyły się sukcesem, udowadniając że 
wymagania funkcjonalne zostały spełnione.

\subsubsection[Spełnienie wymagań jakościowych]{Spełnienie wymagań jakościowych}
Spełnienie pierwszych z wymagań pozafunkcjonalnych: \textit{NFR 01}, \textit{NFR 02} i~\textit{NFR 03} 
możliwe jest dzięki podjęciu odpowiedniej decyzji projektowej dotyczącej wyboru platformy IoT 
opisanej w rozdziale ~\ref{subsection:wp}.
Kolejne wymaganie \textit{NFR 04} dotyczące możliwości uruchamiania GUI 
za pomocą dowolnej przeglądarki,systemu operacyjnego i na dowolnej platformie sprzętowej 
zostało zrealizowane poprzez zaimplementowanie interfejsu użytkownika jako dokumentu html korzystającego z JavaScript. 
Innym wymaganiem funkcjonalnym \textit{NFR 04} jest możliwość łatwego rozszerzenia funkcjonalności o 
kolejne scenariusze testowe. Zrealizowano je poprzez odpowiednią warstwę abstrakcji. Każda konfiguracja 
urządzenia jest realizowana za pomocą osobnego pliku, który zawiera funkcje odpowiedzialne za przeprowadzenie 
testu. Wymagania \textit{NFR 06}, \textit{NFR 07} i~\textit{NFR 10} dotyczą interfejsu użytkownika, a ich 
realizacja udowodniona może być poprzez zrzut ekranu na obrazku~\ref{fig:dahsboard}. 

Podsumowując, wszystkie wymagania jakościowe, poza \textit{NFR 08} i~\textit{NFR 09}, zostały spełnione. 
Niepowodzenie w realizacji wyżej wymienionych wymagań wynika z ograniczonego czasu 
na implementację systemu oraz ich niskiego priorytetu. 




