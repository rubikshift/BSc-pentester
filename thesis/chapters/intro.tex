\section[Wstęp i cel pracy (Michał Krakowiak, Kacper Połom, Jakub Wyka)]{Wstęp i cel pracy}
Wraz z postępującą informatyzacją rośnie ilość zagrożeń czekających na systemy komputerowe.
Sprawia to, że coraz większą wagę przypisuje się odpowiedniemu zabezpieczeniu różnego rodzaju systemów komputerowych.

\textit{Właściwy wstęp do tematu.}

W rozdziale~\ref{chapter:security} zostaną poruszone najważniejsze aspekty dotyczące testowania bezpieczeństwa.
Omówione zostaną podstawowe pojęcia wymagane do zrozumienia kontekstu pracy. Ponadto przytoczone zostają kwestie prawne, żeby zapoznać czytelnika z granicą między działaniami dozwolonymi a przestępstwem.
Poruszona zostanie problematyka związana z bezpieczeństwem fizycznych interfejsów komputera.

Rozdział~\ref{chapter:project} 

Rozdział~\ref{chapter:enviroment}

Rozdział~\ref{chapter:usbkeyboard} skupia się na omówieniu realizacji funkcjonalności symulowania działania klawiatury USB.
Przedstawiony zostanie scenariusz działania funkcjonalności.
Wyjaśnione zostaną mechanizmy, które pozwalają na kreowanie nowych urządzeń konfiguracyjnych, w szczególności tych należących do klasy odpowiedzialnej za interakcje użytkownika.
Przedstawiony zostanie sposób konfiguracji z wykorzystaniem systemu operacyjnego Linux i języka Python. Ponadto poruszone zostaną aspekty, związane z wykryciem przez użytkownika działania realizowanego systemu lub potencjalnego zagrożenia.

Rozdział~\ref{chapter:usbethernet}

Rozdział~\ref{chapter:results}