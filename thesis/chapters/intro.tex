\section[Wstęp i cel pracy (Michał Krakowiak, Kacper Połom, Jakub Wyka)]{Wstęp i cel pracy}
Wraz z postępującą informatyzacją rośnie ilość zagrożeń czekających na systemy komputerowe.
Sprawia to, że coraz większą wagę przypisuje się odpowiedniemu zabezpieczeniu różnego rodzaju systemów komputerowych.

\textit{Właściwy wstęp do tematu i celu.}

W rozdziale~\ref{chapter:security} zostaną poruszone najważniejsze aspekty dotyczące testowania bezpieczeństwa.
Są to m.in.: podstawowe pojęcia wymagane do zrozumienia kontekstu pracy oraz kwestie regulacji prawnych, które  mają przedstawić różnicę między działaniami dozwolonymi a przestępstwem.
Ponadto omawiana jest problematyka bezpieczeństwem fizycznych interfejsów komputera.

Rozdział~\ref{chapter:project} 

Rozdział~\ref{chapter:enviroment}

Rozdział~\ref{chapter:usbkeyboard} skupia się na omówieniu realizacji funkcjonalności symulowania działania klawiatury USB.
Przedstawiony zostanie scenariusz użycia systemu.
Osobne podrozdziały poświęcono na opisanie mechanizmu, które pozwalają na kreowanie nowych urządzeń USB, oraz konfiguracji takiego urządzenia z wykorzystaniem systemu \textit{Linux} i języka programowania \textit{Python}.
Na koniec analizowane są aspekty, związane z wykryciem przez użytkownika działania realizowanego systemu lub potencjalnego zagrożenia.

Rozdział~\ref{chapter:usbethernet}

Rozdział~\ref{chapter:results}