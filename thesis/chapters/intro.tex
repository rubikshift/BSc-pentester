 \section[Wstęp (Michał Krakowiak, Kacper Połom, Jakub Wyka)]{Wstęp}
Wraz z postępującą informatyzacją rośnie ilość zagrożeń na które narażone są współczesne systemy komputerowe.
Sprawia to, że coraz większą wagę przypisuje się odpowiedniemu zabezpieczeniu różnego rodzaju systemów komputerowych.
Wiele publikacji dotyczących tej tematyki traktuje o bezpieczeństwie aplikacji internetowych.
Jednak równie ważne jest testowanie podatności na fizyczne zagrożenia.
W związku z łatwiejszym dostępem do miniaturowej elektroniki w mediach można znaleźć informacje o złośliwych urządzeniach USB np. zaszytych w kablach~\cite{niebezpiecznik}.

\subsection[Cel pracy]{Cel pracy}
Z punktu widzenia osoby przeprowadzającej audyt bezpieczeństwa np. dla dużej korporacji, sprawdzenie bezpieczeństwa każdego pojedyńczego komputera w organizacji może być problematyczne.
Głównym celem projektu była realizacja dedykowanego systemu wspomagającego testowanie bezpieczeństwa systemów komputerowych.
Ze względu na charakter i okoliczności powstania zdecydowano się nazwać go \textit{BSc-pentester}.
Zaprojektowany system korzysta z odpowiednio oprogramowanych akcesoriów, które można przyłączyć do testowanego systemu po przez interfejs USB.
Są one w wstanie symulować działanie klawiatury i karty sieciowej. Możliwe scenariusze testowe, które mogą zostać zrealizowane to:
\begin{itemize}
    \item zatruwanie ustawień sieciowych
    \item przechwytywanie i/lub modyfikowanie ruchu sieciowego
    \item wprowadzanie zaprogramowanej sekwencji klawiszy
\end{itemize}
W celu zapewnienia użyteczności \textit{BSc-pentestera} przyjęliśmy założenie, że powinien on umożliwiać pracę zdalną np. z poziomu przeglądarki internetowej.


\subsection[Zawartość pracy]{Zawartość pracy}
Rozdział~\ref{chapter:security} porusza najważniejsze aspekty dotyczące testowania bezpieczeństwa.
Są to m.in.: podstawowe pojęcia wymagane do zrozumienia kontekstu pracy oraz kwestie regulacji prawnych, które  mają przedstawić różnicę między działaniami dozwolonymi a przestępstwem.
Ponadto omawiana jest problematyka bezpieczeństwem fizycznych interfejsów komputera.

W rozdziale~\ref{chapter:project} zawarty jest projekt \textit{BSc-pentestera}.
Celem jego analitycznej części jest poznanie dziedziny problemu i 
wyróżnienie wymagań względem produktu. Następnie przedstawiany jest faktyczny projektu 
systemu zawierający diagram przypadków użycia, diagram klas oraz projekt interfejsu 
użytkownika.

W rozdziale~\ref{chapter:enviroment} zostanie opisane środowiska testowe. Będą przedstawione cele, oraz podstawowe wymagania. Zostaną również wymienione oraz porównane najpopularniejsze platformy Iot oraz protokoły umożliwiające przesyłanie danych.
Na końcu zostaną poruszone kwestie zaprojektowanej architektury między m.in: sposób komunikacji platformy wykonującej z środowiskiem testowym oraz autoryzacja platform.

Rozdział~\ref{chapter:usbkeyboard} skupia się na omówieniu realizacji funkcjonalności symulowania działania klawiatury USB.
Przedstawiony zostanie scenariusz użycia systemu.
Osobne podrozdziały poświęcono na opisanie mechanizmu, które pozwalają na kreowanie nowych urządzeń USB, oraz konfiguracji takiego urządzenia z wykorzystaniem systemu \textit{Linux} i języka programowania \textit{Python}.
Na koniec analizowane są aspekty, związane z wykryciem przez użytkownika działania \textit{BSc-pentestera} lub potencjalnego zagrożenia.

Celem rozdziału~\ref{chapter:usbethernet} jest omówienie realizacji funkcjonalności 
symulowania działania karty sieciowej. Przedstawiona jest tam ogólna koncepcja 
funkcjonowania urządzenia w danej konfiguracji oraz szczegóły techniczne jej realizacji. 
W dalszej części opisana jest możliwość wykrycia złośliwego urządzenia w powyższej 
konfiguracji oraz dokładny opis realizowanych na nim scenariuszy testowych. 


Rozdział~\ref{chapter:results} ma na celu potwierdzenie, że system działa poprawnie, pokazanie występujących podatności oraz ocenę ryzyka. Rozpatrywanych będzie pięć kategorii zagrożeń (szkody, odtwarzalność, możliwość wykorzystania, dotknięci użytkownicy, wykrywalność), gdzie do każdej zostanie przydzielona ocena.


Z uwagi na stosunkowo niewielką liczbę dostępnych materiałów w języku polskim wykorzystano głównie anglojęzyczne źródła.
Spowodowało to problemy z tłumaczeniem terminologii.
Niektóre terminy nie posiadają polskiego tłumaczenia, w tym przypadku korzystano z oryginalnej nazwy.
Tam gdzie tłumaczenie istnieje, ale mogłoby wprowadzić niejednoznaczność przytacza się dodatkowo angielski odpowiednik.

\subsection[Podział prac]{Podział prac}
Podział zadań, związanych z implementacją, pomiędzy członków zespołu wyglądał następująco:
\begin{itemize}[leftmargin=\parindent]
    \item Michał Krakowiak
    \begin{itemize}
        \item Implementacja symulowania działania klawiatury USB
        \item Przygotowanie metod pozwalających na przeprowadzenie scenariuszy \textit{Mitm} i \textit{Podszycie pod urządzenie HID} (opisane w rozdziałach~\ref{sce:klawiatura} i~\ref{sce:mitm})
    \end{itemize}
    \item Kacper Połom
    \begin{itemize}
        \item Moduł serwera i komunikacji pomiędzy elementami systemu
        \item Wdrożenie środowiska testowego
    \end{itemize}
    \item Jakub Wyka
    \begin{itemize}
        \item Implementacja symulowania działania karty sieciowej USB
        \item Przygotowanie metod pozwalających na przeprowadzenie scenariusza \textit{Pharming} (opisany w rozdziale~\ref{sce:phar}).
    \end{itemize}
\end{itemize}