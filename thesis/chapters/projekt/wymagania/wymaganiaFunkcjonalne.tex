\subsection[Wymagania funkcjonalne (Jakub Wyka)]{Wymagania funkcjonalne}
\begin{table}[H]
    \begin{tabular}{|p{2cm}|p{12cm}|}
        \hline
        \textbf{} & \textbf{Możliwość konfiguracji urządzenia wykonującego jako kartę sieciową} \\
        \hline
        Opis: & Możliwość konfiguracji platformy wykonującej, aby pełniła rolę i była wykrywana przez testowany system jako karta sieciowa. \\
        \hline
        Źródło: & Pentesterzy \\
        \hline
        Priorytet: & krytyczny \\ 
        \hline
    \end{tabular}  
    \label{tab:wym1}
\end{table}
%    \bigskip
\begin{table}[H]
    \begin{tabular}{|p{2cm}|p{12cm}|}
        \hline
        \textbf{} & \textbf{Możliwość konfiguracji urządzenia wykonującego jako klawiaturę} \\
        \hline
        Opis: & Możliwość konfiguracji platformy wykonującej, aby pełniła rolę i była wykrywana przez testowany system jako klawiatura. \\
        \hline
        Źródło: & Pentesterzy \\
        \hline
        Priorytet: & krytyczny \\ 
        \hline
    \end{tabular}  
    \label{tab:wym2}
\end{table}
%    \bigskip
\begin{table}[H]
    \begin{tabular}{|p{2cm}|p{12cm}|}
        \hline
        \textbf{} & \textbf{Możliwość wykonania testu zatruwania DNS} \\
        \hline
        Opis: & Możliwość wykonania testu bezpieczeństwa polegającego na podmianie rekordu DNS. \\
        \hline
        Źródło: & Pentesterzy \\
        \hline
        Priorytet: & krytyczny \\ 
        \hline
    \end{tabular}  
    \label{tab:wym3}
\end{table}
%    \bigskip
\begin{table}[H]
    \begin{tabular}{|p{2cm}|p{12cm}|}
        \hline
        \textbf{} & \textbf{Możliwość wykonania testu przechwytywania pakietów} \\
        \hline
        Opis: & Możliwość wykonania testu bezpieczeństwa polegającego na przechwyceniu pakietów. \\
        \hline
        Źródło: & Pentesterzy \\
        \hline
        Priorytet: & krytyczny \\ 
        \hline
    \end{tabular}  
    \label{tab:wym4}
\end{table}
    %\bigskip
\begin{table}[H]
    \begin{tabular}{|p{2cm}|p{12cm}|}
        \hline
        \textbf{} & \textbf{Możliwość wykonania testu przejęcia działania klawiatury} \\
        \hline
        Opis: & Możliwość wykonania testu bezpieczeństwa polegającego na przechwyceniu kontroli nad klawiaturą, a co za tym idzie nieograniczonych możliowości zaszkodzenia systemowi. \\
        \hline
        Źródło: & Pentesterzy \\
        \hline
        Priorytet: & krytyczny \\ 
        \hline
    \end{tabular}  
    \label{tab:wym5}
\end{table}
%    \bigskip
\begin{table}[H]
    \begin{tabular}{|p{2cm}|p{12cm}|}
        \hline
        \textbf{} & \textbf{Możliwość wyboru rodzaju testu z  poziomu serwera} \\
        \hline
        Opis: & Możliwość wyboru testu do przeprowadzenia z poziomu serwera. \\
        \hline
        Źródło: & Pentesterzy \\
        \hline
        Priorytet: & wysoki \\ 
        \hline
    \end{tabular}  
    \label{tab:wym6}
\end{table}    
%    \bigskip
\begin{table}[H]
    \begin{tabular}{|p{2cm}|p{12cm}|}
        \hline
        \textbf{} & \textbf{Zdalne uruchomienie testu} \\
        \hline
        Opis: & Możliwość rozpoczęcia wykonywania testu z poziomu serwera, który skomunikuje się z platformą testującą.\\
        \hline
        Źródło: & Pentesterzy \\
        \hline
        Priorytet: & wysoki \\ 
        \hline
    \end{tabular}  
    \label{tab:wym7}
\end{table}    
%   \bigskip
\begin{table}[H]
    \begin{tabular}{|p{2cm}|p{12cm}|}
        \hline
        \textbf{} & \textbf{Możliwość wyboru konkretnej platformy egzekucyjnej do przeprowadzenia testu} \\
        \hline
        Opis: & Możliwość wyboru konkretnej platformy egzekucyjnej, na której chcemy przeprowadzić test.\\
        \hline
        Źródło: & Pentesterzy \\
        \hline
        Priorytet: & wysoki \\ 
        \hline
    \end{tabular}  
    \label{tab:wym8}
\end{table}
%    \bigskip
\begin{table}[H]
    \begin{tabular}{|p{2cm}|p{12cm}|}
        \hline
        \textbf{} & \textbf{Możliwość kontroli testów za pomocą GUI na platformie pentestera} \\
        \hline
        Opis: & Możliwość użycia środowiska graficznego do kontroli wykonywania testów.\\
        \hline
        Źródło: & Pentesterzy \\
        \hline
        Priorytet: & średni \\ 
        \hline
    \end{tabular}  
    \label{tab:wym9}
\end{table}
%    \bigskip
\begin{table}[H]
    \begin{tabular}{|p{2cm}|p{12cm}|}
        \hline
        \textbf{} & \textbf{Możliwość raportowania wyników testu} \\
        \hline
        Opis: & Możliwość sporządzenia raportu z wykonania testu.\\
        \hline
        Źródło: & Pentesterzy \\
        \hline
        Priorytet: & krytyczny \\ 
        \hline
    \end{tabular}  
    \label{tab:wym10}
\end{table}
%    \bigskip
\begin{table}[H]
    \begin{tabular}{|p{2cm}|p{12cm}|}
        \hline
        \textbf{} & \textbf{Możliwość zdalnego raportowania wyników testu.} \\
        \hline
        Opis: & Możliwość przesłania sporządzonego raportu na serwer.\\
        \hline
        Źródło: & Pentesterzy \\
        \hline
        Priorytet: & wysoki \\ 
        \hline
    \end{tabular}  
    \label{tab:wym11}
\end{table}
%    \bigskip
\begin{table}[H]
    \begin{tabular}{|p{2cm}|p{12cm}|}
        \hline
        \textbf{} & \textbf{Możliwość przesłania rozkazu wykonania tego samego testu z danymi parametrami na kilka wybranych urządzeń testujących..} \\
        \hline
        Opis: & Możliwość jednoczesnego przesłania rozkazu wykonania tego samego testu z danymi parametrami na kilka wybranych urządzeń testujących..\\
        \hline
        Źródło: & Pentesterzy \\
        \hline
        Priorytet: & średni \\ 
        \hline
    \end{tabular}  
    \label{tab:wym12}
\end{table}