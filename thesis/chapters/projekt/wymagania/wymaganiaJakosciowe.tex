\subsection[Wymagania jakościowe (Jakub Wyka)]{Wymagania jakościowe}
\begin{table}[H]
\begin{tabular}{|p{2cm}|p{12cm}|}
    \hline
    \textbf{} & \textbf{Ochrona danych} \\
    \hline
    Opis: & Ochrona wszystkich danych przechowywanych przez system. \\
    \hline
    Źródło: & Zlecający testy penetracyjne \\
    \hline
    Priorytet: & krytyczny \\ 
    \hline
\end{tabular}  
\label{tab:wymf1}
\end{table}
%\vskip 0.3cm
\begin{table}[H]
\begin{tabular}{|p{2cm}|p{12cm}|}
    \hline
    \textbf{} & \textbf{Obsługa do 100 urządzeń wykonujących testy} \\
    \hline
    Opis: & Jednoczesna obsługa 100 urządzeń wykonujących testy. \\
    \hline
    Źródło: & Pentesterzy \\
    \hline
    Priorytet: & wysoki \\ 
    \hline
\end{tabular}  
\label{tab:wymf2}
\end{table}
%\vskip 0.3cm
\begin{table}[H]
\begin{tabular}{|p{2cm}|p{12cm}|}
    \hline
    \textbf{} & \textbf{Możliwość jednoczesnej pracy 10 pentesterów.} \\
    \hline
    Opis: & Możliwość jednoczesnego zarządzania systemem z 10 platform pentesterów na raz. \\
    \hline
    Źródło: & Pentesterzy \\
    \hline
    Priorytet: & wysoki \\ 
    \hline
\end{tabular}  
\label{tab:wymf3}
\end{table}
%\vskip 0.3cm
\begin{table}[H]
\begin{tabular}{|p{2cm}|p{12cm}|}
    \hline
    \textbf{} & \textbf{Uruchamianie GUI na dowolnej przeglądarce, systemie operacyjnym i platformie sprzętowej.} \\
    \hline
    Opis: & Możliwość uruchomienia GUI niezależnie od rodzaju platformy stosowanej przez pentestera. \\
    \hline
    Źródło: & Pentesterzy \\
    \hline
    Priorytet: & średni \\ 
    \hline
\end{tabular}  
\label{tab:wymf4}
\end{table}
%\vskip 0.3cm
\begin{table}[H]
\begin{tabular}{|p{2cm}|p{12cm}|}
    \hline
    \textbf{} & \textbf{Możliwość łatwego rozszerzenia funkcjonalności o kolejne scenariusze testowe.} \\
    \hline
    Opis: & Możliwość łatwego dodawania nowych scenariuszy testowych. \\
    \hline
    Źródło: & Pentesterzy \\
    \hline
    Priorytet: & średni \\ 
    \hline
\end{tabular}  
\label{tab:wymf5}
\end{table}
%\vskip 0.3cm
\begin{table}[H]
\begin{tabular}{|p{2cm}|p{12cm}|}
    \hline
    \textbf{} & \textbf{Czytelne raporty wykonanego testu.} \\
    \hline
    Opis: & Łatwe do analizy raporty z przeprowadzonych testów. \\
    \hline
    Źródło: & Pentesterzy \\
    \hline
    Priorytet: & średni \\ 
    \hline
\end{tabular}  
\label{tab:wymf6}
\end{table}
%\vskip 0.3cm
\begin{table}[H]
\begin{tabular}{|p{2cm}|p{12cm}|}
    \hline
    \textbf{} & \textbf{Zgodność ze standardami w zakresie GUI.} \\
    \hline
    Opis: & Wykorzystanie standardów w zakresie projektowania GUI pozwoli na intuicyjne korzystanie z systemu i skróci czas wdrożenia. \\
    \hline
    Źródło: & Pentesterzy \\
    \hline
    Priorytet: & średni \\ 
    \hline
\end{tabular}  
\label{tab:wymf7}
\end{table}
%\vskip 0.3cm
\begin{table}[H]
\begin{tabular}{|p{2cm}|p{12cm}|}
    \hline
    \textbf{} & \textbf{Wyświetlanie aktualnego stanu urządzenia Raspberry Pi Zero.} \\
    \hline
    Opis: & Wyświetlanie w dobrze widocznym miejscu informacji o aktualnym stanie urządzenia wykonującego. \\
    \hline
    Źródło: & Pentesterzy \\
    \hline
    Priorytet: & średni \\ 
    \hline
\end{tabular}  
\label{tab:wymf8}
\end{table}
%\vskip 0.3cm
\begin{table}[H]
\begin{tabular}{|p{2cm}|p{12cm}|}
    \hline
    \textbf{} & \textbf{Rozbudowany formularz parametryzacji testu.} \\
    \hline
    Opis: & Wybór parametrów testu przez specjalnie zaprojektowany formularz, aby niektóre z zadań mogły być przeprowadzone przez mniej wyszkolonych pracowników. \\
    \hline
    Źródło: & Pentesterzy \\
    \hline
    Priorytet: & średni \\ 
    \hline
\end{tabular}  
\label{tab:wymf9}
\end{table}
%\vskip 0.3cm
\begin{table}[H]
\begin{tabular}{|p{2cm}|p{12cm}|}
    \hline
    \textbf{} & \textbf{Możliwość samodzielnego wprowadzenia payloadu za pomocą pola tekstowego w gui.} \\
    \hline
    Opis: & Umożliwi to obsługę mniej standardowych wymagań klientów przez bardziej doświadczonych pracowników firmy. \\
    \hline
    Źródło: & Pentesterzy \\
    \hline
    Priorytet: & średni \\ 
    \hline
\end{tabular}  
\label{tab:wymf10}
\end{table}