\subsection[Wizja systemu (Michał Krakowiak)]{Wizja systemu}
Jak określono w rozdziale~\ref{subsec:cel} \textit{BSc-pentester} ma ułatwić pracę pentesterów.
Kluczowym założeniem jest zrealizowanie możliwości przeprowadzania testów z wykorzystaniem fizycznego interfejsu systemu komputerowego.
Wymaga to przygotowania dedykowanego urządzenia, tzw. platformy wykonującej.
Znajdujące się na niej oprogramowanie pozwoli, w zależności do parametrów, wykorzystać działanie mechanizmu \textit{plug and play} i zgłosić się testowanemu systemowi jako obsługiwane akcesorium.
Karta projektu inżynierskiego definiuje, że implementowana będzie możliwość symulowania klawiatury lub karty sieciowej.
Ponadto platforma wykonująca ma pozwolić na realizację takich scenariuszy jak:
\begin{itemize}
    \item zatruwanie ustawień sieciowych,
    \item przechwytywanie i/lub modyfikowanie ruchu sieciowego,
    \item wprowadzanie zaprogramowanej sekwencji klawiszy.
\end{itemize}
Wydaje się nieakceptowalne, żeby system mający ułatwić pracę pentestera wymagał częstych fizycznych interakcji.
Powinna więc istnieć możliwość pracy zdalnej z takim urządzeniem, np. przez dedykowany interfejs użytkownika.
Udogodnieniem będzie dostęp niezależnie od posiadanego systemu operacyjnego np. z poziomu przeglądarki internetowej.
Komputer, z którego pentester będzie przeprowadzał prace, będzie określany platformą pentestera.
Mile widziana przez docelowych użytkowników byłaby także możliwość testowania wielu niezależnych systemów komputerowych jednocześnie.
Wprowadza to kolejny komponent sprzętowy \textit{BSc-pentestera} - serwer, który pozwoli zarządzać aktywnymi platformami wykonującymi.
Musi on obsługiwać ciągłą, niezawodną i bezpieczną komunikację (np. w celu otrzymania raport z testu).
Wskazana będzie także zdalna konfiguracja urządzeń (m.in.: zlecenie wykonania scenariusza, przekazanie parametrów wykonania).
W związku z ofensywnym charakterem testów, którym dedykowany jest system, proces ten można nazwać zbrojeniem, a przekazywane dane ładunkiem \textit{(z ang. payload)}.
Założenie ścisłej współpracy serwera, platform wykonujących i pentestera pozwala zdefiniować obecność dodatkowego komponentu programowego.
Będzie to pewnego rodzaju ekosystem - środowisko testowe, w którym będą się ze sobą komunikować poszczególne moduły \textit{BSc-pentestera}.
Dalszemu projektowi i implementacji podlegają:
\begin{itemize}
    \item oprogramowanie platformy wykonującej,
    \item oprogramowanie serwera,
    \item interfejs użytkownika.
\end{itemize}\newpage
