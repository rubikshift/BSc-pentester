\section*{Abstract}
This thesis is devoted to the realization of the computer system, which will be designed to support security testing.
It was required to plug the system to USB port and simulate the keyboard or network card. 

As part of the design, usage scenarios have been prepared to define the expected way in which each functionality should work.
Stakeholders, users, system components and design requirements were specified. Moreover, class diagrams and user interface design were created. 

The implementation required familiarization with the mechanisms of USB peripherals.
Configuration that allows the test system to detect the keyboard connection and the functionality to program the key sequence to be entered were implemented. 
Further, network adapter configuration was added, which make it possible to control network traffic.
In order to increase the usability of the dedicated system, it is possible to remotely control the devices using server based test environment.

To prove the system meets the requirements specified earlier, security tests using previously developed scenarios were carried out.
The result of this process is documentation defining vulnerabilities in handling USB devices present in tested systems as well as assessment of the risk caused by them.
Additionally, basic possibilities of intrusion detection are contained.

\noindent \textbf{Keywords: } Computer systems security, security testing, penetration test, 
USB, internet of things

\bigbreak

\noindent \textbf{Fields of Science and Technology, in compliance with OECD 
requirements: } Computer and information sciences