\section*{Streszczenie}
Praca poświęcona jest realizacji systemu, którego celem jest wspomaganie przeprowadzania testów bezpieczeństwa.
Wymaganiem była możliwość podłączenia systemu przy użyciu interfejsu USB oraz symulacja działania klawiatury lub karty sieciowej.

W ramach projektowania przygotowano scenariusze użycia, które mają przybliżać oczekiwany sposób działania poszczególnych funkcjonalności.
Wyróżniono udziałowców, użytkowników i komponenty systemu. 
Szczegółowo opracowano wymagania projektowe.
Zrealizowano diagramy klas i projekt interfejsu użytkownika.

Implementacja wymagała zapoznania się z mechanizmami działania urządzeń peryferyjnych USB.
Zrealizowano konfigurację pozwalającą na wykrycie przez testowany system podłączenia klawiatury oraz możliwość zaprogramowania sekwencji klawiszy jaka ma zostać wprowadzona.
Zrealizowano możliwość działania w trybie karty sieciowej oraz kontroli nad obsługiwanym ruchem sieciowym. 
W celu zwiększenia użyteczności dedykowanego systemu zadbano o możliwość zdalnej obsługi urządzeń po przez wdrożenie środowiska testowego.

W celu potwierdzenia realizacji i użyteczności opisywanego systemu przeprowadzono testy bezpieczeństwa realizujące opracowane wcześniej scenariusze.
Rezultatem jest udokumentowanie obecności podatności w testowanych systemach komputerowych i ocena wiążącego się z nimi ryzyka.
Przygotowano podstawowe analizy możliwości wykrycia zagrożeń przez użytkowników danego systemu komputerowego.

\bigbreak

\noindent \textbf{Słowa kluczowe: } Bezpieczeństwo systemów komputerowych, testowanie bezpieczeństwa, test penetracyjny, USB, internet rzeczy

\bigbreak

\noindent \textbf{Dziedzina nauki i techniki, zgodnie z wymogami OECD: } Nauki o~komputerach i~informatyka